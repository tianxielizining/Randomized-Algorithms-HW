\section{Johnson-Lindenstrauss Lemma: Dimension Reduction}
\noindent 1. \blue{Suppose $g$ is a random variable with normal distribution $N(0,1)$.
Prove the following.\\
(a) For odd $n\ge 1$, $\mathbb{E}[g^n]=0$.\\
(b) For even $n \ge 2$, $\mathbb{E}[g^n]\ge 1$.
}

Since $g$ is a random variable having standard normal distribution $N(0,1)$
the probability density function of $g$ is $\frac{1}{\sqrt{2\pi}}e^{\frac{-x^2}{2}}$ for $x\in \mathbb{R}$.
By definition of expected value of a continuous random variable, we have:
\begin{align}
    \nonumber \mathbb{E}[g^n]=\frac{1}{\sqrt{2\pi}}\int_{-\infty }^{+\infty }x^ne^{\frac{-x^2}{2}}\,dx 
\end{align}

We first consider the case for odd $n\ge 1$.
Since $y=e^{\frac{-x^2}{2}}$, $x\in \mathbb{R}$ is an even function
and for odd $n\ge 1$, $y=x^n$, $x\in \mathbb{R}$  is an odd function, we have $y=x^ne^{\frac{-x^2}{2}}$, $x\in \mathbb{R}$  is an odd function.
Thus, $\frac{1}{\sqrt{2\pi}}\int_{-\infty }^{+\infty }x^ne^{\frac{-x^2}{2}}\,dx =0$. 

Then consider the case for even $n \ge 2$.
When $n=2$, $\mathbb{E}[g^2]=\frac{1}{\sqrt{2\pi}}\int_{-\infty }^{+\infty }x^2e^{\frac{-x^2}{2}}\,dx $.
Let $u(x)=x$, then $u'(x)=1$.
Let $v(x)=-e^{\frac{-x^2}{2}}$, then $v'(x)=xe^{\frac{-x^2}{2}}$. Use integration by parts, we have:
\begin{align}
    \nonumber \mathbb{E}[g^2]&=\frac{1}{\sqrt{2\pi}}\int_{-\infty }^{+\infty }x^2e^{\frac{-x^2}{2}}\,dx\\
    \nonumber &=\frac{1}{\sqrt{2\pi}}\int_{-\infty }^{+\infty }x\cdot xe^{\frac{-x^2}{2}}\,dx\\
    \nonumber &=\frac{1}{\sqrt{2\pi}}\left(\left[x\cdot (-e^{\frac{-x^2}{2}})\right]_{-\infty}^{+\infty}-\int_{-\infty }^{+\infty }1\cdot (-e^{\frac{-x^2}{2}})\,dx \right)\\
    \nonumber &=\frac{1}{\sqrt{2\pi}}\left(\left[-xe^{\frac{-x^2}{2}}\right]_{-\infty}^{+\infty}+\int_{-\infty }^{+\infty }e^{\frac{-x^2}{2}}\,dx \right)
\end{align}
By L'Hôpital's rule, we can compute $\left[-xe^{\frac{-x^2}{2}}\right]_{-\infty}^{+\infty}$ as follows:
\begin{align}
    \nonumber &\lim\limits_{x\to+\infty}   \frac{x}{e^{\frac{x^2}{2}}}=  \lim\limits_{x\to+\infty}\frac{1}{xe^{\frac{x^2}{2}}}=0 \implies \lim\limits_{x\to+\infty}   -\frac{x}{e^{\frac{x^2}{2}}}=0\\
    \nonumber &\lim\limits_{x\to-\infty}   \frac{-x}{e^{\frac{x^2}{2}}}=\lim\limits_{x\to-\infty}\frac{-1}{xe^{\frac{x^2}{2}}}=0\\
    \nonumber & \left[-xe^{\frac{-x^2}{2}}\right]_{-\infty}^{+\infty}=\lim\limits_{x\to+\infty}\frac{-x}{e^{\frac{x^2}{2}}}   -\lim\limits_{x\to-\infty} \frac{-x}{e^{\frac{x^2}{2}}}=0
\end{align}
Compute $\int_{-\infty }^{+\infty }e^{\frac{-x^2}{2}}\,dx $ as follows.
\begin{align}
    \nonumber \int_{-\infty }^{+\infty }e^{\frac{-x^2}{2}}\,dx &=\sqrt{(\int_{-\infty }^{+\infty }e^{\frac{-x^2}{2}}\,dx )^2}\\
    \nonumber &=\sqrt{\int_{-\infty }^{+\infty }e^{\frac{-x^2}{2}}\,dx \int_{-\infty }^{+\infty }e^{\frac{-y^2}{2}}\,dy}\\
    \nonumber &=\sqrt{\int_{-\infty }^{+\infty }\int_{-\infty }^{+\infty }e^{\frac{-x^2}{2}}e^{\frac{-y^2}{2}}\,dx\,dy}\\
    \nonumber &=\sqrt{\int_{-\infty }^{+\infty }\int_{-\infty }^{+\infty }e^{\frac{-(x^2+y^2)}{2}}\,dx\,dy}\\
    \nonumber &=\sqrt{\int_{0}^{2\pi}\int_{0 }^{+\infty }\rho e^{\frac{-\rho^2}{2}}\,d\rho\,d\theta }\\
    \nonumber &=\sqrt{-2\pi\cdot\left[e^{\frac{-\rho^2}{2}}\right]_{0}^{+\infty}}\\
    \nonumber &=\sqrt{-2\pi\cdot(0-1)}\\
    \nonumber &=\sqrt{2\pi}
\end{align}
We can get the same result by $\frac{1}{\sqrt{2\pi}}e^{\frac{-x^2}{2}}$ is the probability density function of $N(0,1)$.
$\int_{-\infty }^{+\infty }\frac{1}{\sqrt{2\pi}}e^{\frac{-x^2}{2}}\,dx=1 \implies \int_{-\infty }^{+\infty }e^{\frac{-x^2}{2}}\,dx=\sqrt{2\pi}$.

Thus,
\begin{align}
    \nonumber \mathbb{E}[g^2]=&\frac{1}{\sqrt{2\pi}}(\left[-xe^{\frac{-x^2}{2}}\right]_{-\infty}^{+\infty}+\int_{-\infty }^{+\infty }e^{\frac{-x^2}{2}}\,dx )=1
\end{align}
For even $n > 2$, use integration by parts again.
Let $f(x)=x^{n-1}$, then $f'(x)=(n-1)x^{n-2}$. Let $h(x)=-e^{\frac{-x^2}{2}} $, then $h'(x)= xe^{\frac{-x^2}{2}}$
\begin{align}
    \nonumber \mathbb{E}[g^{n+2}]&=\frac{1}{\sqrt{2\pi}}\int_{-\infty }^{+\infty }x^{n+2}e^{\frac{-x^2}{2}}\,dx\\
    \nonumber &=\frac{1}{\sqrt{2\pi}}\int_{-\infty }^{+\infty }x^{n+1}\cdot x e^{\frac{-x^2}{2}}\,dx\\
    \nonumber &=\frac{1}{\sqrt{2\pi}}\left(\left[x^{n-1}\cdot (-e^{\frac{-x^2}{2}})\right]_{-\infty}^{+\infty}-\int_{-\infty }^{+\infty }(n-1)x^{n-2}\cdot (-e^{\frac{-x^2}{2}})\,dx \right)\\
    \nonumber &=\frac{1}{\sqrt{2\pi}}\left(\left[-x^{n-1}e^{\frac{-x^2}{2}})\right]_{-\infty}^{+\infty}+\int_{-\infty }^{+\infty }(n-1)x^{n-2}e^{\frac{-x^2}{2}}\,dx \right)
\end{align}
By L'Hôpital's rule, we can compute $\left[-x^{n-1}e^{\frac{-x^2}{2}}\right]_{-\infty}^{+\infty}=0$ in the same way.
Thus,
\begin{align}
    \nonumber \mathbb{E}[g^{n+2}]=\frac{1}{\sqrt{2\pi}}\int_{-\infty }^{+\infty }(n-1)x^{n-2}e^{\frac{-x^2}{2}}\,dx 
    =(n-1)\mathbb{E}[g^{n}]
\end{align} 
Then we have:
\begin{align}
    \nonumber \frac{\mathbb{E}[g^{n+2}]}{\mathbb{E}[g^{n}]}=n-1
\end{align}
Since $n>2$ and $n$ is even, we have $n-1\ge 3$.
Then $\mathbb{E}[g^{n+2}]>\mathbb{E}[g^{n}]$.
We have shown that $\mathbb{E}[g^{2}]=1$, then we have for even $n \ge 2$, $\mathbb{E}[g^n]\ge 1$.\\
2.\blue{Suppose $\gamma _j$'s are independent uniform $\{-1,1\}$-random variables 
and $g_j$'s are independent random variables, each having normal distribution $N(0,1)$. 
Suppose $v_j$'s are real numbers, and define $X:=(\sum_{j}\gamma _jv_j)^2$ and $\widehat{X}:=(\sum_{j}g_jv_j)^2$.
Show that for all integers $n\ge 1$,$E[X^n]\le E[\widehat{X}^n]$.}
\begin{align}
    \nonumber \mathbb{E}[X^n]&=\mathbb{E}\left[\left(\sum_{j=1}^{m}\gamma _jv_j\right)^{2n}\right]\\
    \nonumber &=\mathbb{E}\left[\sum_{k_1+k_2+...+k_m=2n}\binom{2n}{k_1,k_2,...,k_m}\prod_{t=1}^{m}(\gamma_jv_j)^{k_t} \right]\\
    \nonumber &=\sum_{k_1+k_2+...+k_m=2n}\binom{2n}{k_1,k_2,...,k_m}\prod_{t=1}^{m}v_j^{k_t}\mathbb{E}\left[\gamma_j^{k_t}\right]
\end{align}
\begin{align}
    \nonumber \mathbb{E}[\widehat{X}^n]&=\mathbb{E}\left[\left(\sum_{j=1}^{m}g_jv_j\right)^{2n}\right]\\
    \nonumber &=\mathbb{E}\left[\sum_{k_1+k_2+...+k_m=2n}\binom{2n}{k_1,k_2,...,k_m}\prod_{t=1}^{m}(g_jv_j)^{k_t} \right]\\
    \nonumber &=\sum_{k_1+k_2+...+k_m=2n}\binom{2n}{k_1,k_2,...,k_m}\prod_{t=1}^{m}v_j^{k_t}\mathbb{E}\left[g_j^{k_t}\right]
\end{align}

Since the fact that for odd $n\ge 1$, $\mathbb{E}[\gamma^n]=\mathbb{E}[g^n]=0$, 
for even ‎$n\ge 2$, $\mathbb{E}[\gamma^n]\le\mathbb{E}[g^n]$, we have that for all integers $n\ge 1$, $\mathbb{E}[X^n] \le \mathbb{E}[\widehat{X}^n]$.

\noindent 3.\blue{Suppose $Z$ is a random variable having normal distribution $N(0,\nu^2)$. Compute $\mathbb{E}\left[e^{tZ^2}\right]$. 
For what values of $t$ is your expression valid?}
\begin{align}
    \nonumber \mathbb{E}\left[ e^{tZ^2} \right]&=\int_{-\infty }^{+\infty }\frac{1}{\nu \sqrt{2\pi}}e^{tx^2}e^{\frac{-x^2}{2\nu ^2}}\,dx\\
    \nonumber &=\frac{1}{\nu \sqrt{2\pi}}\int_{-\infty }^{+\infty }e^{tx^2}e^{\frac{-x^2}{2\nu^2}}\,dx\\
    \nonumber &=\frac{1}{\nu \sqrt{2\pi}}\int_{-\infty }^{+\infty }e^{(t-\frac{1}{2\nu^2})x^2}\,dx\\
    \nonumber &=\frac{1}{\nu \sqrt{2\pi}}\sqrt{\int_{-\infty }^{+\infty }\int_{-\infty }^{+\infty }e^{(t-\frac{1}{2\nu^2})(x^2+y^2)}\,dx\,dy}\\
    \nonumber &=\frac{1}{\nu \sqrt{2\pi}}\sqrt{ \int_{0}^{2\pi}\int_{0 }^{+\infty }\rho e^{(t-\frac{1}{2\nu^2})\rho^2}\,d\rho\,d\theta   }\\
    \nonumber &=\frac{1}{\nu \sqrt{2\pi}}\sqrt{ 2\pi\cdot\frac{\nu^2}{2t\nu^2-1}\left[e^{ (t-\frac{1}{2\nu^2})\rho^2}\right]_{0}^{+\infty} }
\end{align}
When $t-\frac{1}{2\nu^2}<0\implies t<\frac{1}{2\nu^2}$, we have,
\begin{align}
    \nonumber \left[e^{ (t-\frac{1}{2\nu^2})\rho^2}\right]_{0}^{+\infty}=-1
\end{align}
Then,
\begin{align}
    \nonumber &=\frac{1}{\nu \sqrt{2\pi}}\sqrt{ 2\pi\cdot\frac{\nu^2}{2t\nu^2-1}\left[e^{ (t-\frac{1}{2\nu^2})\rho^2}\right]_{0}^{+\infty} }\\
    \nonumber &=\frac{1}{\nu \sqrt{2\pi}}\sqrt{2\pi\cdot\frac{\nu^2}{1-2t\nu^2}}\\
    \nonumber &=(1-2t\nu^2)^{-\frac{1}{2}}
\end{align}
4.\blue{In this question, we investigate if Johnson-Lindenstrauss Lemma can preserve area.
}\\
\blue{(a)Suppose the distances between three points are preserved with multiplicative error $\epsilon$. 
Is the area of the corresponding triangle also always preserved with multiplicative error $O(\epsilon)$, 
or even some constant multiplicative error?}

