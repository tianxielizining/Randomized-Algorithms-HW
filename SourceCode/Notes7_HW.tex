\section{Johnson-Lindenstrauss Lemma: Dimension Reduction}
\noindent 1. \blue{Suppose $g$ is a random variable with normal distribution $N(0,1)$.
Prove the following.\\
(a) For odd $n\ge 1$, $\mathbb{E}[g^n]=0$.\\
(b) For even $n \ge 2$, $\mathbb{E}[g^n]\ge 1$.
}

Since $g$ is a random variable having standard normal distribution $N(0,1)$
the probability density function of $g$ is $\frac{1}{\sqrt{2\pi}}e^{\frac{-x^2}{2}}$ for $x\in \mathbb{R}$.
By definition of expected value of a continuous random variable, we have:
\begin{align}
    \nonumber \mathbb{E}[g^n]=\frac{1}{\sqrt{2\pi}}\int_{-\infty }^{+\infty }x^ne^{\frac{-x^2}{2}}\,dx 
\end{align}

We first consider the case for odd $n\ge 1$.
Since $y=e^{\frac{-x^2}{2}}$, $x\in \mathbb{R}$ is an even function
and for odd $n\ge 1$, $y=x^n$, $x\in \mathbb{R}$  is an odd function, we have $y=x^ne^{\frac{-x^2}{2}}$, $x\in \mathbb{R}$  is an odd function.
Thus, $\frac{1}{\sqrt{2\pi}}\int_{-\infty }^{+\infty }x^ne^{\frac{-x^2}{2}}\,dx =0$. 

Then consider the case for even $n \ge 2$.
When $n=2$, $\mathbb{E}[g^2]=\frac{1}{\sqrt{2\pi}}\int_{-\infty }^{+\infty }x^2e^{\frac{-x^2}{2}}\,dx $.
Let $u(x)=x$, then $u'(x)=1$.
Let $v(x)=-e^{\frac{-x^2}{2}}$, then $v'(x)=xe^{\frac{-x^2}{2}}$. Use integration by parts, we have:
\begin{align}
    \nonumber \mathbb{E}[g^2]&=\frac{1}{\sqrt{2\pi}}\int_{-\infty }^{+\infty }x^2e^{\frac{-x^2}{2}}\,dx\\
    \nonumber &=\frac{1}{\sqrt{2\pi}}\int_{-\infty }^{+\infty }x\cdot xe^{\frac{-x^2}{2}}\,dx\\
    \nonumber &=\frac{1}{\sqrt{2\pi}}\left(\left[x\cdot (-e^{\frac{-x^2}{2}})\right]_{-\infty}^{+\infty}-\int_{-\infty }^{+\infty }1\cdot (-e^{\frac{-x^2}{2}})\,dx \right)\\
    \nonumber &=\frac{1}{\sqrt{2\pi}}\left(\left[-xe^{\frac{-x^2}{2}}\right]_{-\infty}^{+\infty}+\int_{-\infty }^{+\infty }e^{\frac{-x^2}{2}}\,dx \right)
\end{align}
By L'Hôpital's rule, we can compute $\left[-xe^{\frac{-x^2}{2}}\right]_{-\infty}^{+\infty}$ as follows:
\begin{align}
    \nonumber &\lim\limits_{x\to+\infty}   \frac{x}{e^{\frac{x^2}{2}}}=  \lim\limits_{x\to+\infty}\frac{1}{xe^{\frac{x^2}{2}}}=0 \implies \lim\limits_{x\to+\infty}   -\frac{x}{e^{\frac{x^2}{2}}}=0\\
    \nonumber &\lim\limits_{x\to-\infty}   \frac{-x}{e^{\frac{x^2}{2}}}=\lim\limits_{x\to-\infty}\frac{-1}{xe^{\frac{x^2}{2}}}=0\\
    \nonumber & \left[-xe^{\frac{-x^2}{2}}\right]_{-\infty}^{+\infty}=\lim\limits_{x\to+\infty}\frac{-x}{e^{\frac{x^2}{2}}}   -\lim\limits_{x\to-\infty} \frac{-x}{e^{\frac{x^2}{2}}}=0
\end{align}
Compute $\int_{-\infty }^{+\infty }e^{\frac{-x^2}{2}}\,dx $ as follows.
\begin{align}
    \nonumber \int_{-\infty }^{+\infty }e^{\frac{-x^2}{2}}\,dx &=\sqrt{(\int_{-\infty }^{+\infty }e^{\frac{-x^2}{2}}\,dx )^2}\\
    \nonumber &=\sqrt{\int_{-\infty }^{+\infty }e^{\frac{-x^2}{2}}\,dx \int_{-\infty }^{+\infty }e^{\frac{-y^2}{2}}\,dy}\\
    \nonumber &=\sqrt{\int_{-\infty }^{+\infty }\int_{-\infty }^{+\infty }e^{\frac{-x^2}{2}}e^{\frac{-y^2}{2}}\,dx\,dy}\\
    \nonumber &=\sqrt{\int_{-\infty }^{+\infty }\int_{-\infty }^{+\infty }e^{\frac{-(x^2+y^2)}{2}}\,dx\,dy}\\
    \nonumber &=\sqrt{\int_{0}^{2\pi}\int_{0 }^{+\infty }\rho e^{\frac{-\rho^2}{2}}\,d\rho\,d\theta }\\
    \nonumber &=\sqrt{-2\pi\cdot\left[e^{\frac{-\rho^2}{2}}\right]_{0}^{+\infty}}\\
    \nonumber &=\sqrt{-2\pi\cdot(0-1)}\\
    \nonumber &=\sqrt{2\pi}
\end{align}
We can get the same result by $\frac{1}{\sqrt{2\pi}}e^{\frac{-x^2}{2}}$ is the probability density function of $N(0,1)$.
$\int_{-\infty }^{+\infty }\frac{1}{\sqrt{2\pi}}e^{\frac{-x^2}{2}}\,dx=1 \implies \int_{-\infty }^{+\infty }e^{\frac{-x^2}{2}}\,dx=\sqrt{2\pi}$.

Thus,
\begin{align}
    \nonumber \mathbb{E}[g^2]=&\frac{1}{\sqrt{2\pi}}(\left[-xe^{\frac{-x^2}{2}}\right]_{-\infty}^{+\infty}+\int_{-\infty }^{+\infty }e^{\frac{-x^2}{2}}\,dx )=1
\end{align}
For even $n > 2$, use integration by parts again.
Let $f(x)=x^{n-1}$, then $f'(x)=(n-1)x^{n-2}$. Let $h(x)=-e^{\frac{-x^2}{2}} $, then $h'(x)= xe^{\frac{-x^2}{2}}$
\begin{align}
    \nonumber \mathbb{E}[g^{n+2}]&=\frac{1}{\sqrt{2\pi}}\int_{-\infty }^{+\infty }x^{n+2}e^{\frac{-x^2}{2}}\,dx\\
    \nonumber &=\frac{1}{\sqrt{2\pi}}\int_{-\infty }^{+\infty }x^{n+1}\cdot x e^{\frac{-x^2}{2}}\,dx\\
    \nonumber &=\frac{1}{\sqrt{2\pi}}\left(\left[x^{n-1}\cdot (-e^{\frac{-x^2}{2}})\right]_{-\infty}^{+\infty}-\int_{-\infty }^{+\infty }(n-1)x^{n-2}\cdot (-e^{\frac{-x^2}{2}})\,dx \right)\\
    \nonumber &=\frac{1}{\sqrt{2\pi}}\left(\left[-x^{n-1}e^{\frac{-x^2}{2}})\right]_{-\infty}^{+\infty}+\int_{-\infty }^{+\infty }(n-1)x^{n-2}e^{\frac{-x^2}{2}}\,dx \right)
\end{align}
By L'Hôpital's rule, we can compute $\left[-x^{n-1}e^{\frac{-x^2}{2}}\right]_{-\infty}^{+\infty}=0$ in the same way.
Thus,
\begin{align}
    \nonumber \mathbb{E}[g^{n+2}]=\frac{1}{\sqrt{2\pi}}\int_{-\infty }^{+\infty }(n-1)x^{n-2}e^{\frac{-x^2}{2}}\,dx 
    =(n-1)\mathbb{E}[g^{n}]
\end{align} 
Then we have:
\begin{align}
    \nonumber \frac{\mathbb{E}[g^{n+2}]}{\mathbb{E}[g^{n}]}=n-1
\end{align}
Since $n>2$ and $n$ is even, we have $n-1\ge 3$.
Then $\mathbb{E}[g^{n+2}]>\mathbb{E}[g^{n}]$.
We have shown that $\mathbb{E}[g^{2}]=1$, then we have for even $n \ge 2$, $\mathbb{E}[g^n]\ge 1$.\\
2.\blue{Suppose $\gamma _j$'s are independent uniform $\{-1,1\}$-random variables 
and $g_j$'s are independent random variables, each having normal distribution $N(0,1)$. 
Suppose $v_j$'s are real numbers, and define $X:=(\sum_{j}\gamma _jv_j)^2$ and $\widehat{X}:=(\sum_{j}g_jv_j)^2$.
Show that for all integers $n\ge 1$,$E[X^n]\le E[\widehat{X}^n]$.}
\begin{align}
    \nonumber \mathbb{E}[X^n]&=\mathbb{E}\left[\left(\sum_{j=1}^{m}\gamma _jv_j\right)^{2n}\right]\\
    \nonumber &=\mathbb{E}\left[\sum_{k_1+k_2+...+k_m=2n}\binom{2n}{k_1,k_2,...,k_m}\prod_{t=1}^{m}(\gamma_jv_j)^{k_t} \right]\\
    \nonumber &=\sum_{k_1+k_2+...+k_m=2n}\binom{2n}{k_1,k_2,...,k_m}\prod_{t=1}^{m}v_j^{k_t}\mathbb{E}\left[\gamma_j^{k_t}\right]
\end{align}
\begin{align}
    \nonumber \mathbb{E}[\widehat{X}^n]&=\mathbb{E}\left[\left(\sum_{j=1}^{m}g_jv_j\right)^{2n}\right]\\
    \nonumber &=\mathbb{E}\left[\sum_{k_1+k_2+...+k_m=2n}\binom{2n}{k_1,k_2,...,k_m}\prod_{t=1}^{m}(g_jv_j)^{k_t} \right]\\
    \nonumber &=\sum_{k_1+k_2+...+k_m=2n}\binom{2n}{k_1,k_2,...,k_m}\prod_{t=1}^{m}v_j^{k_t}\mathbb{E}\left[g_j^{k_t}\right]
\end{align}
Fix non-negative integers $k_1,k_2,...,k_m$, and compare  $\prod_{t=1}^{m}v_j^{k_t}\mathbb{E}\left[\gamma_j^{k_t}\right]$ and $\prod_{t=1}^{m}v_j^{k_t}\mathbb{E}\left[g_j^{k_t}\right]$ as follows.
If there exites $k_t$, $t\in\{1,2,...,m\}$, such that $k_t$ is odd, and $k_t>=1$, then, $\mathbb{E}[\gamma^{k_t}]=\mathbb{E}[g^k_t]=0$, we have $\prod_{t=1}^{m}v_j^{k_t}\mathbb{E}\left[\gamma_j^{k_t}\right]=\prod_{t=1}^{m}v_j^{k_t}\mathbb{E}\left[g_j^{k_t}\right]=0$.
Otherwise, for all $k_t$s, $t\in\{1,2,...,m\}$, $k_t$s are even, which implies for all real number $v_j$s, $v_j^{k_t}\ge 0$. 
Since $k_1+k_2+...+k_m=2n$, it's impossible that all $k_t$s, $t\in \{1,2,...,m\}$ are 0.
For $k_t=0$, we have $\mathbb{E}[\gamma^{k_t}]=\mathbb{E}[g^{k_t}]$.
For even $k_t\ge 2$, we have $\mathbb{E}[\gamma^{k_t}]\le\mathbb{E}[g^{k_t}]$. 
Then, $\prod_{t=1}^{m}v_j^{k_t}\mathbb{E}\left[\gamma_j^{k_t}\right]\le \prod_{t=1}^{m}v_j^{k_t}\mathbb{E}\left[g_j^{k_t}\right]$.

\noindent 3.\blue{Suppose $Z$ is a random variable having normal distribution $N(0,\nu^2)$. Compute $\mathbb{E}\left[e^{tZ^2}\right]$. 
For what values of $t$ is your expression valid?}
\begin{align}
    \nonumber \mathbb{E}\left[ e^{tZ^2} \right]&=\int_{-\infty }^{+\infty }\frac{1}{\nu \sqrt{2\pi}}e^{tx^2}e^{\frac{-x^2}{2\nu ^2}}\,dx
    %\nonumber &=\frac{1}{\nu \sqrt{2\pi}}\int_{-\infty }^{+\infty }e^{tx^2}e^{\frac{-x^2}{2\nu^2}}\,dx\\
    %\nonumber &=\frac{1}{\nu \sqrt{2\pi}}\int_{-\infty }^{+\infty }e^{(t-\frac{1}{2\nu^2})x^2}\,dx\\
    %\nonumber &=\frac{1}{\nu \sqrt{2\pi}}\sqrt{\int_{-\infty }^{+\infty }\int_{-\infty }^{+\infty }e^{(t-\frac{1}{2\nu^2})(x^2+y^2)}\,dx\,dy}\\
    %\nonumber &=\frac{1}{\nu \sqrt{2\pi}}\sqrt{ \int_{0}^{2\pi}\int_{0 }^{+\infty }\rho e^{(t-\frac{1}{2\nu^2})\rho^2}\,d\rho\,d\theta   }
    %\nonumber &=\frac{1}{\nu \sqrt{2\pi}}\sqrt{ 2\pi\cdot\frac{\nu^2}{2t\nu^2-1}\left[e^{ (t-\frac{1}{2\nu^2})\rho^2}\right]_{0}^{+\infty} }
\end{align}

When $t-\frac{1}{2\nu^2}<0\implies t<\frac{1}{2\nu^2}$, we have,
%\begin{align}
 %   \nonumber \left[e^{ (t-\frac{1}{2\nu^2})\rho^2}\right]_{0}^{+\infty}=-1
%\end{align}
%Then,
\begin{align}
    \nonumber \mathbb{E}\left[ e^{tZ^2} \right]&=\int_{-\infty }^{+\infty }\frac{1}{\nu \sqrt{2\pi}}e^{tx^2}e^{\frac{-x^2}{2\nu ^2}}\,dx\\
    \nonumber &=\frac{1}{\nu \sqrt{2\pi}}\int_{-\infty }^{+\infty }e^{tx^2}e^{\frac{-x^2}{2\nu^2}}\,dx\\
    \nonumber &=\frac{1}{\nu \sqrt{2\pi}}\int_{-\infty }^{+\infty }e^{(t-\frac{1}{2\nu^2})x^2}\,dx\\
    \nonumber &=\frac{1}{\nu \sqrt{2\pi}}\sqrt{\int_{-\infty }^{+\infty }\int_{-\infty }^{+\infty }e^{(t-\frac{1}{2\nu^2})(x^2+y^2)}\,dx\,dy}\\
    \nonumber &=\frac{1}{\nu \sqrt{2\pi}}\sqrt{ \int_{0}^{2\pi}\int_{0 }^{+\infty }\rho e^{(t-\frac{1}{2\nu^2})\rho^2}\,d\rho\,d\theta   }\\
    \nonumber &=\frac{1}{\nu \sqrt{2\pi}}\sqrt{ 2\pi\cdot\frac{\nu^2}{2t\nu^2-1}\left[e^{ (t-\frac{1}{2\nu^2})\rho^2}\right]_{0}^{+\infty} }\\
    \nonumber &=\frac{1}{\nu \sqrt{2\pi}}\sqrt{2\pi\cdot\frac{\nu^2}{1-2t\nu^2}}\\
    \nonumber &=(1-2t\nu^2)^{-\frac{1}{2}}
\end{align}

When $t-\frac{1}{2\nu^2}\ge0\implies t\ge\frac{1}{2\nu^2}$, %we have,
%\begin{align}
    %\nonumber \left[e^{ (t-\frac{1}{2\nu^2})\rho^2}\right]_{0}^{+\infty}=+\infty
%\end{align}
%Thus 
$\frac{1}{\nu \sqrt{2\pi}}e^{tx^2}e^{\frac{-x^2}{2\nu ^2}}$ is not integrable on $(-\infty, +\infty )$.% when $t-\frac{1}{2\nu^2}>0$.

%When $t-\frac{1}{2\nu^2}=0 \implies \rho e^{(t-\frac{1}{2\nu^2})\rho^2}=\rho$. $\rho$ is not integrable on $[0, +\infty )$.
%Since when $t-\frac{1}{2\nu^2}=0 \implies 2t\nu^2-1=0$, which is a  denominator of our expression, $t-\frac{1}{2\nu^2}$ can not be equal to $0$.

In conclution, $\mathbb{E}\left[ e^{tZ^2} \right]=(1-2t\nu^2)^{-\frac{1}{2}}$ when $t<\frac{1}{2\nu^2}$.

\noindent4.\blue{In this question, we investigate if Johnson-Lindenstrauss Lemma can preserve area.
}

\blue{(a) Suppose the distances between three points are preserved with multiplicative error $\epsilon$. 
Is the area of the corresponding triangle also always preserved with multiplicative error $O(\epsilon)$, 
or even some constant multiplicative error?}

The area  of the corresponding triangle is not preserved.
The area of a triangle is related to not only the size of the side but also the height of the side.
For example, let $A=(0,100)$, $B=(0,0)$, $C=(1,0)$, $\epsilon=0.01$, we can compute $100<||AC||<101$.
Let $A^\prime=(0,100)$, $B^\prime=(0,0)$, $C^\prime=(0,-1)$.
In that setting, we have that $||AB||=||A^\prime B^\prime||=100$, $||BC||=||B^\prime C^\prime||=1$, $||A^\prime C^\prime||=101\implies (1-\epsilon)||AC||<||A^\prime C^\prime||<(1+\epsilon)||AC||$.
In another word, the distances between the three points are preserved with multiplicative error $\epsilon$.
However the area turned to be $0$, which means the area  of the corresponding triangle is not preserved.

\blue{(b) Suppose $u$ and $v$ are mutually orthogonal unit vectors. 
Observe that the vectors $u$ and $v$ together with the origin 
form a right-angled isosceles triangle with area $\frac{1}{2}$. 
Suppose the lengths of the triangle are distorted with multiplicative error at most $\epsilon$. 
What is the multiplicative error for the area of the triangle?}

Let $f$ be the function that can preserve distances between two points with multiplicative error $\epsilon$.
That is for two points $x$ and $y$, we have:
\begin{align}
    \nonumber |||x-y||-||f(x)-f(y)||| \le \epsilon ||x-y|| \implies
    (1-\epsilon)||x-y||\le||f(x)-f(y)||\le(1+\epsilon)||x-y||
\end{align}
Let $\theta=\angle(f(u),f(v))$.
\begin{align}
    \nonumber &||f(u)-f(v)||^2=||f(u)||^2+||f(v)||^2-2f(u)\cdot f(v)=||f(u)||^2+||f(v)||^2-2\cos\theta\\
    \nonumber &\implies 2 \cos\theta =||f(u)||^2+||f(v)||^2-||f(u)-f(v)||^2\\
    \nonumber &\implies 2 \cos\theta\in (1\pm\epsilon)^2||u||^2+(1\pm\epsilon)^2||v||^2-(1\pm \epsilon)^2||u-v||^2\\
    \nonumber &\implies 2 \cos\theta\in  2(1\pm\epsilon)^2-2(1\pm \epsilon)^2\\
    \nonumber &\implies \cos\theta\in (1\pm\epsilon)^2-(1\pm \epsilon)^2\\
    \nonumber &\implies \cos\theta \in [-4\epsilon,4\epsilon]\\
    \nonumber &\implies (\cos\theta)^2\le16\epsilon^2
\end{align}

Let $\epsilon \le \frac{1}{4}$.
When  $\cos\theta=4\epsilon$, we have $||f(u)||=(1+\epsilon) ||u||$, $||f(v)||=(1+\epsilon) ||v||$, $||f(u)-f(v)||=(1-\epsilon) ||u-v||$.
When  $\cos\theta=-4\epsilon$, we have $||f(u)||=(1-\epsilon )||u||$, $||f(v)||=(1-\epsilon) ||v||$, $||f(u)-f(v)||=(1+\epsilon) ||u-v||$.
Since $(\sin\theta)^2+(\cos\theta)^2=1$, we have $\sin\theta \ge \sqrt{1-16\epsilon^2}$.

Let $S^{\prime}$ be the area of the triangle under function $f$, we have:
$S^{\prime}=||f(u)||||f(v)||sin\theta \ge (1-\epsilon)^2 \cdot \sqrt{1-16\epsilon^2}$, 
$S^{\prime}=||f(u)||||f(v)||sin\theta \le (1+\epsilon)^2$.
Let $S$ be the area of original triangle, $S=\frac{1}{2}$.
Thus we have $||S^{\prime}-S||\le (2\epsilon^2+4\epsilon+1)S$.

\blue{(c) Suppose a set $V$ of $n$ points are given in Euclidean space $\mathbb{R}^n$. 
Let $0<\epsilon<1$. 
Give a randomized algorithm that produces a low-dimensional mapping $f:V\rightarrow \mathbb{R}^T$ such that
the areas of all triangles formed from the $n$ points are preserved with multiplicative error $\epsilon$.
What is the value $T$ for your mapping? Please give the exact number and do not use big $O$ notation. 
}

From (a), we can know, although the distances between three points are preserved, the areas of triangle is not preserved.
The main reason is that we cannot preserve the height of each sides.
From (b), we know that for right-angled isosceles triangle, if we preserve the distances between three vertices, we can preserve the area of the triangle.
Then, we can use right-angled isosceles triangle to preserve the height of any triangle as follows.
For any triangle, w.l.o.g, Suppose the triangle is composed of three points $X,Y,Z$.
In order to preserve the height of each side, we have to form a right-angled isosceles triangle by letting the height be a right-angled side.
For each side of the triangle, add two points as follows.
Use side $XY$ as an example.
The first point is the projection point of $Z$ on line $XY$, use $P_{XY1}$ to denote the point.
Use $P_{XY2}$ to denote the second point, there are two constrains for ponit $P_{XY2}$,
one is that $P_{XY2}$ should on  line $XY$, the other is that $||P_{XY2}-P_{XY1}||=||Z-P_{XY1}||$.
By the same method, we can find the two points for side $XZ$ and side $YZ$.

Since there are $n$ points, we can use the $n$ points to construct $\binom{n}{3}$ triangles.
For each side of each triangle, we add two additional points, thus, there are totally $n+6\binom{n}{3}$ points.
Let $V^{\prime}$ be the set of the $n+6\binom{n}{3}$ points.
By Johnson-Lindenstrauss Lemma, for any $0<\epsilon<1$, suppose $U$ is a set of $n$ points, 
we have there exist a mapping $f: U\rightarrow \mathbb{R}^T$, where $T=O(\frac{\log n}{\epsilon^{\prime 2}})$, 
such that $\sqrt{1-\epsilon^{\prime}}||x-y||<||f(x)-f(y)||<\sqrt{1+\epsilon^{\prime}}||x-y||$.
Since we have to guarantee the multiplicative error with in $\epsilon$, by (b), we should let $\epsilon^{\prime}=$
