\section{Johnson-Lindenstrauss Lemma: Dimension Reduction}
\noindent 1. \blue{Suppose $g$ is a random variable with normal distribution $N(0,1)$.
Prove the following.\\
(a) For odd $n\ge 1$, $\mathbb{E}[g^n]=0$.\\
(b) For even $n \ge 2$, $\mathbb{E}[g^n]\ge 1$.
}

Since $g$ is a random variable having standard normal distribution $N(0,1)$
the probability density function of $g$ is $\frac{1}{\sqrt{2\pi}}e^{\frac{-x^2}{2}}$ for $x\in \mathbb{R}$.
By definition of expected value of a continuous random variable, we have:
\begin{align}
    \nonumber \mathbb{E}[g^n]=\frac{1}{\sqrt{2\pi}}\int_{-\infty }^{+\infty }x^ne^{\frac{-x^2}{2}}\,dx 
\end{align}

We first consider the case for odd $n\ge 1$.
Since $y=e^{\frac{-x^2}{2}}$, $x\in \mathbb{R}$ is an even function
and for odd $n\ge 1$, $y=x^n$, $x\in \mathbb{R}$  is an odd function, we have $y=x^ne^{\frac{-x^2}{2}}$, $x\in \mathbb{R}$  is an odd function.
Thus, $\frac{1}{\sqrt{2\pi}}\int_{-\infty }^{+\infty }x^ne^{\frac{-x^2}{2}}\,dx =0$. 

Then consider the case for even $n \ge 2$.
When $n=2$, $\mathbb{E}[g^2]=\frac{1}{\sqrt{2\pi}}\int_{-\infty }^{+\infty }x^2e^{\frac{-x^2}{2}}\,dx $.
Let $u(x)=x$, then $u'(x)=1$.
Let $v(x)=-e^{\frac{-x^2}{2}}$, then $v'(x)=xe^{\frac{-x^2}{2}}$. Use integration by parts, we have:
\begin{align}
    \nonumber \mathbb{E}[g^2]&=\frac{1}{\sqrt{2\pi}}\int_{-\infty }^{+\infty }x^2e^{\frac{-x^2}{2}}\,dx\\
    \nonumber &=\frac{1}{\sqrt{2\pi}}\int_{-\infty }^{+\infty }x\cdot xe^{\frac{-x^2}{2}}\,dx\\
    \nonumber &=\frac{1}{\sqrt{2\pi}}\left(\left[x\cdot (-e^{\frac{-x^2}{2}})\right]_{-\infty}^{+\infty}-\int_{-\infty }^{+\infty }1\cdot (-e^{\frac{-x^2}{2}})\,dx \right)\\
    \nonumber &=\frac{1}{\sqrt{2\pi}}\left(\left[-xe^{\frac{-x^2}{2}}\right]_{-\infty}^{+\infty}+\int_{-\infty }^{+\infty }e^{\frac{-x^2}{2}}\,dx \right)
\end{align}
By L'Hôpital's rule, we can compute $\left[-xe^{\frac{-x^2}{2}}\right]_{-\infty}^{+\infty}$ as follows:
\begin{align}
    \nonumber &\lim\limits_{x\to+\infty}   \frac{x}{e^{\frac{x^2}{2}}}=  \lim\limits_{x\to+\infty}\frac{1}{xe^{\frac{x^2}{2}}}=0 \implies \lim\limits_{x\to+\infty}   -\frac{x}{e^{\frac{x^2}{2}}}=0\\
    \nonumber &\lim\limits_{x\to-\infty}   \frac{-x}{e^{\frac{x^2}{2}}}=\lim\limits_{x\to-\infty}\frac{-1}{xe^{\frac{x^2}{2}}}=0\\
    \nonumber & \left[-xe^{\frac{-x^2}{2}}\right]_{-\infty}^{+\infty}=\lim\limits_{x\to+\infty}\frac{-x}{e^{\frac{x^2}{2}}}   -\lim\limits_{x\to-\infty} \frac{-x}{e^{\frac{x^2}{2}}}=0
\end{align}
Compute $\int_{-\infty }^{+\infty }e^{\frac{-x^2}{2}}\,dx $ as follows.
\begin{align}
    \nonumber \int_{-\infty }^{+\infty }e^{\frac{-x^2}{2}}\,dx &=\sqrt{(\int_{-\infty }^{+\infty }e^{\frac{-x^2}{2}}\,dx )^2}\\
    \nonumber &=\sqrt{\int_{-\infty }^{+\infty }e^{\frac{-x^2}{2}}\,dx \int_{-\infty }^{+\infty }e^{\frac{-y^2}{2}}\,dy}\\
    \nonumber &=\sqrt{\int_{-\infty }^{+\infty }\int_{-\infty }^{+\infty }e^{\frac{-x^2}{2}}e^{\frac{-y^2}{2}}\,dx\,dy}\\
    \nonumber &=\sqrt{\int_{-\infty }^{+\infty }\int_{-\infty }^{+\infty }e^{\frac{-(x^2+y^2)}{2}}\,dx\,dy}\\
    \nonumber &=\sqrt{\int_{0}^{2\pi}\int_{0 }^{+\infty }\rho e^{\frac{-\rho^2}{2}}\,d\rho\,d\theta }\\
    \nonumber &=\sqrt{-2\pi\cdot\left[e^{\frac{-\rho^2}{2}}\right]_{0}^{+\infty}}\\
    \nonumber &=\sqrt{-2\pi\cdot(0-1)}\\
    \nonumber &=\sqrt{2\pi}
\end{align}
We can get the same result by $\frac{1}{\sqrt{2\pi}}e^{\frac{-x^2}{2}}$ is the probability density function of $N(0,1)$.
$\int_{-\infty }^{+\infty }\frac{1}{\sqrt{2\pi}}e^{\frac{-x^2}{2}}\,dx=1 \implies \int_{-\infty }^{+\infty }e^{\frac{-x^2}{2}}\,dx=\sqrt{2\pi}$.

Thus,
\begin{align}
    \nonumber \mathbb{E}[g^2]=&\frac{1}{\sqrt{2\pi}}(\left[-xe^{\frac{-x^2}{2}}\right]_{-\infty}^{+\infty}+\int_{-\infty }^{+\infty }e^{\frac{-x^2}{2}}\,dx )=1
\end{align}
For even $n > 2$, use integration by parts again.
Let $f(x)=x^{n-1}$, then $f'(x)=(n-1)x^{n-2}$. Let $h(x)=-e^{\frac{-x^2}{2}} $, then $h'(x)= xe^{\frac{-x^2}{2}}$
\begin{align}
    \nonumber \mathbb{E}[g^{n+2}]&=\frac{1}{\sqrt{2\pi}}\int_{-\infty }^{+\infty }x^{n+2}e^{\frac{-x^2}{2}}\,dx\\
    \nonumber &=\frac{1}{\sqrt{2\pi}}\int_{-\infty }^{+\infty }x^{n+1}\cdot x e^{\frac{-x^2}{2}}\,dx\\
    \nonumber &=\frac{1}{\sqrt{2\pi}}\left(\left[x^{n-1}\cdot (-e^{\frac{-x^2}{2}})\right]_{-\infty}^{+\infty}-\int_{-\infty }^{+\infty }(n-1)x^{n-2}\cdot (-e^{\frac{-x^2}{2}})\,dx \right)\\
    \nonumber &=\frac{1}{\sqrt{2\pi}}\left(\left[-x^{n-1}e^{\frac{-x^2}{2}})\right]_{-\infty}^{+\infty}+\int_{-\infty }^{+\infty }(n-1)x^{n-2}e^{\frac{-x^2}{2}}\,dx \right)
\end{align}
By L'Hôpital's rule, we can compute $\left[-x^{n-1}e^{\frac{-x^2}{2}}\right]_{-\infty}^{+\infty}=0$ in the same way.
Thus,
\begin{align}
    \nonumber \mathbb{E}[g^{n+2}]=\frac{1}{\sqrt{2\pi}}\int_{-\infty }^{+\infty }(n-1)x^{n-2}e^{\frac{-x^2}{2}}\,dx 
    =(n-1)\mathbb{E}[g^{n}]
\end{align} 
Then we have:
\begin{align}
    \nonumber \frac{\mathbb{E}[g^{n+2}]}{\mathbb{E}[g^{n}]}=n-1
\end{align}
Since $n>2$ and $n$ is even, we have $n-1\ge 3$.
Then $\mathbb{E}[g^{n+2}]>\mathbb{E}[g^{n}]$.
We have shown that $\mathbb{E}[g^{2}]=1$, then we have for even $n \ge 2$, $\mathbb{E}[g^n]\ge 1$.\\
2.\blue{Suppose $\gamma _j$'s are independent uniform $\{-1,1\}$-random variables 
and $g_j$'s are independent random variables, each having normal distribution $N(0,1)$. 
Suppose $v_j$'s are real numbers, and define $X:=(\sum_{j}\gamma _jv_j)^2$ and $\widehat{X}:=(\sum_{j}g_jv_j)^2$.
Show that for all integers $n\ge 1$,$E[X^n]\le E[\widehat{X}^n]$.}
\begin{align}
    \nonumber \mathbb{E}[X^n]&=\mathbb{E}\left[\left(\sum_{j=1}^{m}\gamma _jv_j\right)^{2n}\right]\\
    \nonumber &=\mathbb{E}\left[\sum_{k_1+k_2+...+k_m=2n}\binom{2n}{k_1,k_2,...,k_m}\prod_{t=1}^{m}(\gamma_jv_j)^{k_t} \right]\\
    \nonumber &=\sum_{k_1+k_2+...+k_m=2n}\binom{2n}{k_1,k_2,...,k_m}\prod_{t=1}^{m}v_j^{k_t}\mathbb{E}\left[\gamma_j^{k_t}\right]
\end{align}
\begin{align}
    \nonumber \mathbb{E}[\widehat{X}^n]&=\mathbb{E}\left[\left(\sum_{j=1}^{m}g_jv_j\right)^{2n}\right]\\
    \nonumber &=\mathbb{E}\left[\sum_{k_1+k_2+...+k_m=2n}\binom{2n}{k_1,k_2,...,k_m}\prod_{t=1}^{m}(g_jv_j)^{k_t} \right]\\
    \nonumber &=\sum_{k_1+k_2+...+k_m=2n}\binom{2n}{k_1,k_2,...,k_m}\prod_{t=1}^{m}v_j^{k_t}\mathbb{E}\left[g_j^{k_t}\right]
\end{align}
Fix non-negative integers $k_1,k_2,...,k_m$, and compare  $\prod_{t=1}^{m}v_j^{k_t}\mathbb{E}\left[\gamma_j^{k_t}\right]$ and $\prod_{t=1}^{m}v_j^{k_t}\mathbb{E}\left[g_j^{k_t}\right]$ as follows.
If there exites $k_t$, $t\in\{1,2,...,m\}$, such that $k_t$ is odd, and $k_t>=1$, then, $\mathbb{E}[\gamma^{k_t}]=\mathbb{E}[g^k_t]=0$, we have $\prod_{t=1}^{m}v_j^{k_t}\mathbb{E}\left[\gamma_j^{k_t}\right]=\prod_{t=1}^{m}v_j^{k_t}\mathbb{E}\left[g_j^{k_t}\right]=0$.
Otherwise, for all $k_t$s, $t\in\{1,2,...,m\}$, $k_t$s are even, which implies for all real number $v_j$s, $v_j^{k_t}\ge 0$. 
Since $k_1+k_2+...+k_m=2n$, it's impossible that all $k_t$s, $t\in \{1,2,...,m\}$ are 0.
For $k_t=0$, we have $\mathbb{E}[\gamma^{k_t}]=\mathbb{E}[g^{k_t}]$.
For even $k_t\ge 2$, we have $\mathbb{E}[\gamma^{k_t}]\le\mathbb{E}[g^{k_t}]$. 
Then, $\prod_{t=1}^{m}v_j^{k_t}\mathbb{E}\left[\gamma_j^{k_t}\right]\le \prod_{t=1}^{m}v_j^{k_t}\mathbb{E}\left[g_j^{k_t}\right]$.

\noindent 3.\blue{Suppose $Z$ is a random variable having normal distribution $N(0,\nu^2)$. Compute $\mathbb{E}\left[e^{tZ^2}\right]$. 
For what values of $t$ is your expression valid?}
\begin{align}
    \nonumber \mathbb{E}\left[ e^{tZ^2} \right]&=\int_{-\infty }^{+\infty }\frac{1}{\nu \sqrt{2\pi}}e^{tx^2}e^{\frac{-x^2}{2\nu ^2}}\,dx
    %\nonumber &=\frac{1}{\nu \sqrt{2\pi}}\int_{-\infty }^{+\infty }e^{tx^2}e^{\frac{-x^2}{2\nu^2}}\,dx\\
    %\nonumber &=\frac{1}{\nu \sqrt{2\pi}}\int_{-\infty }^{+\infty }e^{(t-\frac{1}{2\nu^2})x^2}\,dx\\
    %\nonumber &=\frac{1}{\nu \sqrt{2\pi}}\sqrt{\int_{-\infty }^{+\infty }\int_{-\infty }^{+\infty }e^{(t-\frac{1}{2\nu^2})(x^2+y^2)}\,dx\,dy}\\
    %\nonumber &=\frac{1}{\nu \sqrt{2\pi}}\sqrt{ \int_{0}^{2\pi}\int_{0 }^{+\infty }\rho e^{(t-\frac{1}{2\nu^2})\rho^2}\,d\rho\,d\theta   }
    %\nonumber &=\frac{1}{\nu \sqrt{2\pi}}\sqrt{ 2\pi\cdot\frac{\nu^2}{2t\nu^2-1}\left[e^{ (t-\frac{1}{2\nu^2})\rho^2}\right]_{0}^{+\infty} }
\end{align}

When $t-\frac{1}{2\nu^2}<0\implies t<\frac{1}{2\nu^2}$, we have,
%\begin{align}
 %   \nonumber \left[e^{ (t-\frac{1}{2\nu^2})\rho^2}\right]_{0}^{+\infty}=-1
%\end{align}
%Then,
\begin{align}
    \nonumber \mathbb{E}\left[ e^{tZ^2} \right]&=\int_{-\infty }^{+\infty }\frac{1}{\nu \sqrt{2\pi}}e^{tx^2}e^{\frac{-x^2}{2\nu ^2}}\,dx\\
    \nonumber &=\frac{1}{\nu \sqrt{2\pi}}\int_{-\infty }^{+\infty }e^{tx^2}e^{\frac{-x^2}{2\nu^2}}\,dx\\
    \nonumber &=\frac{1}{\nu \sqrt{2\pi}}\int_{-\infty }^{+\infty }e^{(t-\frac{1}{2\nu^2})x^2}\,dx\\
    \nonumber &=\frac{1}{\nu \sqrt{2\pi}}\sqrt{\int_{-\infty }^{+\infty }\int_{-\infty }^{+\infty }e^{(t-\frac{1}{2\nu^2})(x^2+y^2)}\,dx\,dy}\\
    \nonumber &=\frac{1}{\nu \sqrt{2\pi}}\sqrt{ \int_{0}^{2\pi}\int_{0 }^{+\infty }\rho e^{(t-\frac{1}{2\nu^2})\rho^2}\,d\rho\,d\theta   }\\
    \nonumber &=\frac{1}{\nu \sqrt{2\pi}}\sqrt{ 2\pi\cdot\frac{\nu^2}{2t\nu^2-1}\left[e^{ (t-\frac{1}{2\nu^2})\rho^2}\right]_{0}^{+\infty} }\\
    \nonumber &=\frac{1}{\nu \sqrt{2\pi}}\sqrt{2\pi\cdot\frac{\nu^2}{1-2t\nu^2}}\\
    \nonumber &=(1-2t\nu^2)^{-\frac{1}{2}}
\end{align}

When $t-\frac{1}{2\nu^2}\ge0\implies t\ge\frac{1}{2\nu^2}$, %we have,
%\begin{align}
    %\nonumber \left[e^{ (t-\frac{1}{2\nu^2})\rho^2}\right]_{0}^{+\infty}=+\infty
%\end{align}
%Thus 
$\frac{1}{\nu \sqrt{2\pi}}e^{tx^2}e^{\frac{-x^2}{2\nu ^2}}$ is not integrable on $(-\infty, +\infty )$.% when $t-\frac{1}{2\nu^2}>0$.

%When $t-\frac{1}{2\nu^2}=0 \implies \rho e^{(t-\frac{1}{2\nu^2})\rho^2}=\rho$. $\rho$ is not integrable on $[0, +\infty )$.
%Since when $t-\frac{1}{2\nu^2}=0 \implies 2t\nu^2-1=0$, which is a  denominator of our expression, $t-\frac{1}{2\nu^2}$ can not be equal to $0$.

In conclution, $\mathbb{E}\left[ e^{tZ^2} \right]=(1-2t\nu^2)^{-\frac{1}{2}}$ when $t<\frac{1}{2\nu^2}$.

\noindent4.\blue{In this question, we investigate if Johnson-Lindenstrauss Lemma can preserve area.
}

\blue{(a) Suppose the distances between three points are preserved with multiplicative error $\epsilon$. 
Is the area of the corresponding triangle also always preserved with multiplicative error $O(\epsilon)$, 
or even some constant multiplicative error?}

The area of the corresponding triangle is not preserved.
The area of a triangle is related to not only the size of the side but also the height of the corresponding side.
Although the size of each side is preserved, the height of the corresponding side may turn to $0$.
Thus the area of the corresponding triangle turns to $0$.

Suppose there is a triangle $ABC$ and $||AB||=1$, $||BC||=t$, $||AC||=\sqrt{1+t^2}$, 
w.l.o.g, set $t\ge 1$ and the value of $t$ would be chosen later according to the value of $\epsilon$.
Let $S$ be the area of triangle $ABC$, then $S=\frac{t}{2}$.

Let $A^{\prime}B^{\prime}C^{\prime}$ be the triangle after transformation.
Assume after transformation, 
$||A^\prime B^\prime ||=1$, $||B^\prime C^\prime ||=t$, $||A^\prime C^\prime ||=t+1$. 
Here, $A^{\prime}$, $B^\prime$, $C^\prime$ are in the same line.
%$A^\prime=(0,-1)$, $B^\prime(0,0)$, $C^\prime(0,t)$.
Let $S^\prime$ be the area of triangle $A^\prime B^\prime C^\prime$,
then we have $S^\prime=0$.
Since the distances between three points are preserved with multiplicative error $\epsilon$ , 
we have,
\begin{align}
    \nonumber ||A^\prime C^\prime||\le (1+\epsilon)||AC||\implies t+1 \le (1+\epsilon) \sqrt{1+t^2}\implies (\epsilon^2+2\epsilon)t^2-2t+(\epsilon^2+2\epsilon)\ge 0
\end{align}
If $\Delta<0\implies \epsilon > \sqrt{2}-1$, the only constrain of $t$ is $t\ge 1$.
Otherwise, 
since $t\ge1$, we have $t\ge \frac{1+\sqrt{1-(\epsilon^2+2\epsilon)^2}}{\epsilon^2+2\epsilon}$.
Under the above setting, we have that 
the distances between the three points are preserved with multiplicative error $\epsilon$,
the area of the corresponding triangle is not preserved.

\noindent\textbf{Another approach: }
Suppose there is a triangle $ABC$ and $||BC||=a$, $||AC||=b$, $||AB||=c$ where $a$, $b$, and $c$ are positive real numbers.
Since $ABC$ is a triangle, we can have, $a+b>c$.
%Let $S$ be the area of triangle $ABC$, then,
%\begin{align}
 %   \nonumber S=\sqrt{\frac{a+b+c}{2}\cdot \frac{a+b-c}{2}\cdot \frac{a-b+c}{2}\cdot \frac{-a+b+c}{2}}
%end{align}
Let $A^{\prime}B^{\prime}C^{\prime}$ be the triangle after transformation.
Suppose after transformation,  $||B^\prime C^\prime ||=a^\prime$, $||A^\prime C^\prime ||=b^\prime$, $||A^\prime B^\prime ||=c^\prime$. 
Since the distances between three points are preserved with multiplicative error $\epsilon$, we have
\begin{align}
    \nonumber (1-\epsilon)a &\le a^{\prime}\le (1+\epsilon)a\\
    \nonumber (1-\epsilon)b &\le b^{\prime}\le (1+\epsilon)b\\
    \nonumber (1-\epsilon)c &\le c^{\prime}\le (1+\epsilon)c
\end{align} 
%Let $S^{\prime}$ be the area of triangle $A^{\prime}B^{\prime}C^{\prime}$.
If $a+b-c \le 3\epsilon$ and after transformation, $a^{\prime}=a-\epsilon$, $b^{\prime}=b-\epsilon$, $c^{\prime}=c+\epsilon$,
then we have,
$a^\prime+b^\prime-c^\prime=a+b-c-3\epsilon \le 0 \implies a^\prime+b^\prime\le c^\prime$, which means $A^{\prime}B^{\prime}C^{\prime}$ is not a triangle.
Thus we cannot preserve the area of triangle $ABC$.

\blue{(b) Suppose $u$ and $v$ are mutually orthogonal unit vectors. 
Observe that the vectors $u$ and $v$ together with the origin 
form a right-angled isosceles triangle with area $\frac{1}{2}$. 
Suppose the lengths of the triangle are distorted with multiplicative error at most $\epsilon$. 
What is the multiplicative error for the area of the triangle?}

Let $F$ be any function that can preserve distances between two points with multiplicative error $\epsilon$.
That is for two points $x$ and $y$, we have:
\begin{align}
    \nonumber |||x-y||-||F(x)-F(y)||| \le \epsilon ||x-y|| \implies
    (1-\epsilon)||x-y||\le||F(x)-F(y)||\le(1+\epsilon)||x-y||
\end{align}
Let $\theta=\angle(f(u),f(v))$.
\begin{align}
    \nonumber &||F(u)-F(v)||^2=||F(u)||^2+||F(v)||^2-2\langle F(u), F(v)\rangle\\
    \nonumber &\implies 2\langle F(u), F(v)\rangle =||F(u)||^2+||F(v)||^2-||F(u)-F(v)||^2\\
    \nonumber &\implies 2\langle F(u), F(v)\rangle\in (1\pm\epsilon)^2||u||^2+(1\pm\epsilon)^2||v||^2-(1\pm \epsilon)^2||u-v||^2\\
    \nonumber &\implies 2\langle F(u), F(v)\rangle\in  2(1\pm\epsilon)^2-2(1\pm \epsilon)^2\\
    \nonumber &\implies \langle F(u), F(v)\rangle\in (1\pm\epsilon)^2-(1\pm \epsilon)^2\\
    \nonumber &\implies \langle F(u), F(v)\rangle \in [-4\epsilon,4\epsilon]\\
    \nonumber &\implies ||F(u)||||F(v)|| \cos \theta \in [-4\epsilon,4\epsilon]\\
    \nonumber &\implies \cos \theta \in \left[\frac{-4\epsilon}{||F(u)||||F(v)||},\frac{4\epsilon}{||F(u)||||F(v)||}\right]\\
    \nonumber &\implies (\cos \theta)^2 \le \frac{16\epsilon^2}{||F(u)||^2||F(v)||^2}
\end{align}
We should ensure that $\frac{4\epsilon}{||F(u)||||F(v)||}\le1 \implies \epsilon \le \frac{||F(u)||||F(v)||}{4}$.
Since $(1-\epsilon) ||u||\le||F(u)||\le (1+\epsilon) ||u||$, $(1-\epsilon) ||v||\le||F(v)||\le (1+\epsilon) ||v||$,
we should let $ \epsilon \le \frac{(1-\epsilon)^2}{4}\implies \epsilon \in (0,3-2\sqrt{2}]$.
%When  $\cos\theta=4\epsilon$, we have $||f(u)||=(1+\epsilon) ||u||$, $||f(v)||=(1+\epsilon) ||v||$, $||f(u)-f(v)||=(1-\epsilon) ||u-v||$.
%When  $\cos\theta=-4\epsilon$, we have $||f(u)||=(1-\epsilon )||u||$, $||f(v)||=(1-\epsilon) ||v||$, $||f(u)-f(v)||=(1+\epsilon) ||u-v||$.
Since $(\sin\theta)^2+(\cos\theta)^2=1$ and the fact that $\theta\le\pi$, we have $\sin\theta \ge \sqrt{1-\frac{16\epsilon^2}{||F(u)||^2||F(v)||^2}}$.

Let $S^{\prime}$ be the area of the triangle under function $f$, we have:
\begin{align}
    \nonumber S^{\prime}=\frac{1}{2}||F(u)||||F(v)||\sin\theta &\ge \frac{1}{2}||F(u)||||F(v)||\cdot \sqrt{1-\frac{16\epsilon}{||F(u)||^2||F(v)||^2}}\\
    \nonumber &=\frac{1}{2}\sqrt{||F(u)||^2||F(v)||^2-16\epsilon^2}\\
    \nonumber &\ge \frac{1}{2}\sqrt{(1-\epsilon)^4-16\epsilon^2}\\
    \nonumber S^{\prime}=\frac{1}{2}||F(u)||||F(v)||\sin\theta &\le \frac{1}{2}(1+\epsilon)^2
\end{align}

Let $S$ be the area of original triangle, $S=\frac{1}{2}$.
Then we have 
when $S^{\prime}\le S$, $S-S^{\prime}\le (1-\sqrt{(1-\epsilon)^4-16\epsilon^2})S$,
otherwise, $S^{\prime}-S \le ((1+\epsilon)^2-1)S$.
Thus we have:
\begin{align}
    \nonumber |S^{\prime}-S|\le S \cdot \max(1-\sqrt{(1-\epsilon)^4-16\epsilon^2}, (1+\epsilon)^2-1)=(1-\sqrt{(1-\epsilon)^4-16\epsilon^2})S
\end{align}
Since 
\begin{align}
    \nonumber 1-\sqrt{(1-\epsilon)^4-16\epsilon^2} 
    &= 1-\sqrt{\epsilon^4-4\epsilon^3+6\epsilon^2-4\epsilon+1-16\epsilon^2}\\
    \nonumber &\le 1-\sqrt{-4\epsilon^2+6\epsilon^2-4\epsilon+1-16\epsilon^2}\\
    \nonumber &= 1-\sqrt{-14\epsilon^2-4\epsilon+1}
\end{align}
After the above transformation, we should let $\epsilon \in (0,\frac{3\sqrt{2}-2}{14}]$.
Then we can conclude that 
\begin{align}
    \nonumber |S^{\prime}-S|\le (1-\sqrt{-14\epsilon^2-4\epsilon+1})S
\end{align}

\blue{(c) Suppose a set $V$ of $n$ points are given in Euclidean space $\mathbb{R}^n$. 
Let $0<\epsilon<1$. 
Give a randomized algorithm that produces a low-dimensional mapping $f:V\rightarrow \mathbb{R}^T$ such that
the areas of all triangles formed from the $n$ points are preserved with multiplicative error $\epsilon$.
What is the value $T$ for your mapping? Please give the exact number and do not use big $O$ notation. 
}

%From (a), we can know, although the distances between three points are preserved, the areas of triangle may not be preserved.
%The main reason is that we cannot preserve the height of each sides. In another word, there is a small angle in the triangle, we cannot preserve the angle and the angle turns to $0^{\circ}$, which causes the area turns to $0$.
%From (b), we know that for a specific right-angled isosceles triangle, if we preserve the distances between three vertices, we can preserve the area of the triangle.
The main idea is as follows:
First prove that for any right-angled isosceles triangle, if the distances between three vertices are preserved, the area of the triangle is preserved.
Second, for any triangle that fromed by the $n$ points, find a right-angled isosceles triangle such that the two triangles are in the same plane.
Then find a linear mapping that can preserve the distances between any points.
Since the fact that if two triangles lie in the same plane in $\mathbb{R}^n$, 
then under a linear mapping their areas have the same multiplicative error, we can conclude that the areas of all the triangles are preserved.

By the same method of question (b), we can get that for any right-angled isosceles triangle, if we preserve the distances between three vertices, we can preserve the area of the triangle.
The process is as follows.
W.l.o.g, let points $A$,$B$,$C$ be three vertices of a right-angled isosceles triangle, let angle $A$ be the right angle.
Let $||A-B||=h$, $||A-C||=h$, $||B-C||=\sqrt{2}h$ where $h>0$.
%Use vector $u$ to denote side $AB$, use vector $v$ to denote side $AC$.
Let $F$ be any function that can preserve distances between two points with multiplicative error $\epsilon^{\prime}$.
That is for two points $x$ and $y$, we have:
\begin{align}
    \nonumber |||x-y||-||F(x)-F(y)||| \le \epsilon^{\prime} ||x-y|| \implies
    (1-\epsilon^{\prime})||x-y||\le||F(x)-F(y)||\le(1+\epsilon^{\prime})||x-y||
\end{align}
\begin{align}
    \nonumber &||F(B)-F(C)||^2=||F(A)-F(B)||^2+||F(A)-F(C)||^2-2||F(A)-F(B)||||F(A)-F(C)||\cos A \\
    \nonumber &\implies 2||F(A)-F(B)||||F(A)-F(C)||\cos A =||F(A)-F(B)||^2+||F(A)-F(C)||^2-||F(B)-F(C)||^2\\
    \nonumber &\implies 2 ||F(A)-F(B)||||F(A)-F(C)||\cos A\in (1\pm\epsilon^{\prime})^2||A-B||^2+(1\pm\epsilon^{\prime})^2||A-C||^2-(1\pm \epsilon^{\prime})^2||B-C||^2\\
    \nonumber &\implies 2 ||F(A)-F(B)||||F(A)-F(C)||\cos A\in  (2(1\pm\epsilon^{\prime})^2-2(1\pm \epsilon^{\prime})^2)h^2\\
    \nonumber &\implies ||F(A)-F(B)||||F(A)-F(C)||\cos A\in ((1\pm\epsilon^{\prime})^2-(1\pm \epsilon^{\prime})^2)h^2\\
    \nonumber &\implies ||F(A)-F(B)||||F(A)-F(C)||\cos A \in [-4\epsilon^{\prime} h^2,4\epsilon^{\prime} h^2]\\
    \nonumber &\implies \cos A \in \left[\frac{-4\epsilon^{\prime}h^2}{||F(A)-F(B)||||F(A)-F(C)||},\frac{4\epsilon^{\prime}h^2}{||F(A)-F(B)||||F(A)-F(C)||}\right]\\
    \nonumber &\implies (\cos A)^2 \le \frac{16\epsilon^{\prime 2}h^4}{||F(A)-F(B)||^2||F(A)-F(C)||^2}
\end{align}

We should ensure that $\frac{4\epsilon^{\prime}h^2}{||F(A)-F(B)||||F(A)-F(C)||}\le1 \implies \epsilon^{\prime} \le \frac{||F(A)-F(B)||||F(A)-F(C)||}{4h^2}$.
Since $(1-\epsilon^{\prime}) ||A-B||\le||F(A)-F(B)||\le (1+\epsilon^{\prime}) ||A-B||$, $(1-\epsilon^{\prime}) ||A-C||\le||F(A)-F(C)||\le (1+\epsilon^{\prime}) ||A-C||$,
we should let $ \epsilon^{\prime} \le \frac{(1-\epsilon^{\prime})^2h^2}{4h^2}\implies \epsilon^{\prime} \in (0,3-2\sqrt{2}]$.
Since $(\sin A)^2+(\cos A)^2=1$ and the fact that $A\le\pi$, we have $\sin A \ge \sqrt{1- \frac{16\epsilon^{\prime 2}h^4}{||F(A)-F(B)||^2||F(A)-F(C)||^2}}$.
Let $S^{\prime}$ be the area of the triangle under function $f$, we have:
\begin{align}
    \nonumber S^{\prime}&=\frac{1}{2}||F(A)-F(B)||||F(A)-F(C)||\sin A \\
    \nonumber &\ge \frac{1}{2}||F(A)-F(B)||||F(A)-F(C)||\cdot \sqrt{1- \frac{16\epsilon^{\prime 2}h^4}{||F(A)-F(B)||^2||F(A)-F(C)||^2}}\\
    \nonumber &=\frac{1}{2}\sqrt{||F(A)-F(B)||^2||F(A)-F(C)||^2-16\epsilon^{\prime 2}h^4}\\
    \nonumber &\ge \frac{1}{2}\sqrt{(1-\epsilon^{\prime})^4h^4-16\epsilon^{\prime 2}h^4}\\
    \nonumber &=\frac{1}{2}h^2\sqrt{(1-\epsilon^{\prime})^4-16\epsilon^{\prime 2}}\\
    \nonumber S^{\prime}&=\frac{1}{2}||F(u)||||F(v)||\sin A \le \frac{1}{2}(1+\epsilon^{\prime})^2h^2
\end{align}

Let $S$ be the area of original triangle, $S=\frac{1}{2}h^2$.
Then we have 
when $S^{\prime}\le S$, $S-S^{\prime}\le (1-\sqrt{(1-\epsilon^{\prime})^4-16\epsilon^{\prime 2}})S$,
otherwise, $S^{\prime}-S \le ((1+\epsilon^{\prime})^2-1)S$.
Thus we have:
\begin{align}
    \nonumber |S^{\prime}-S|\le S \cdot \max(1-\sqrt{(1-\epsilon^{\prime})^4-16\epsilon^{\prime 2}}, (1+\epsilon^{\prime})^2-1)=(1-\sqrt{(1-\epsilon^{\prime})^4-16\epsilon^{\prime 2}})S
\end{align}
Since 
\begin{align}
    \nonumber 1-\sqrt{(1-\epsilon^{\prime})^4-16\epsilon^{\prime 2}} 
    &= 1-\sqrt{\epsilon^{\prime 4}-4\epsilon^{\prime 3}+6\epsilon^{\prime 2}-4\epsilon^{\prime}+1-16\epsilon^{\prime 2}}\\
    \nonumber &\le 1-\sqrt{-4\epsilon^{\prime 2}+6\epsilon^{\prime 2}-4\epsilon^{\prime}+1-16\epsilon^{\prime 2}}\\
    \nonumber &= 1-\sqrt{-14\epsilon^{\prime 2}-4\epsilon^{\prime}+1}
\end{align}
After the above transformation, we should let $\epsilon^{\prime} \in (0,\frac{3\sqrt{2}-2}{14}]$.
Then we can conclude that 
\begin{align}
    \nonumber |S^{\prime}-S|\le (1-\sqrt{-14\epsilon^{\prime 2}-4\epsilon^{\prime}+1})S
\end{align}

For any triangle, w.l.o.g, suppose the triangle is composed of three points $X,Y,Z$.
%In order to preserve the height of each side, we have to form a right-angled isosceles triangle by letting the height be a right-angled side.
Use triangle $XYZ$ to construct a right-angled isosceles triangle as follows. 
The main idea is to add at most two points to triangle $XYZ$.
Arbitrarily choose a side of the triangle, add two points as follows.
Suppose we choose side $XY$.
The first point is the projection point of $Z$ on line $XY$, use $P_{XY1}$ to denote the point.
Use $P_{XY2}$ to denote the second point, there are two constrains for ponit $P_{XY2}$,
one is that $P_{XY2}$ should on  line $XY$, the other is that $||P_{XY2}-P_{XY1}||=||Z-P_{XY1}||$.
There are two points that can satisfy the two constrains, choose one of the two points arbitrarily.
Thus triangle $ZP_{XY1}P_{XY2}$ is a right-angled isosceles triangle.
And the area of triangle $ZP_{XY1}P_{XY2}$ is preserved. 
It is a fact that a line and a point not on that line can determine a unique plane.
Use $L$ to denote the plane that is determined by point $Z$ and line $XY$.
Since $P_{XY1}$ and $P_{XY2}$ are on line $XY$, we can get 
triangle $ZP_{XY1}P_{XY2}$ and triangle $XYZ$ are in the same plane $L$.

Since the fact that if two triangles lie in the same plane in $\mathbb{R}^n$, 
then under a linear mapping their areas have the same multiplicative error.
The next step is to find a linear mapping $f:\mathbb{R}^n\rightarrow\mathbb{R}^T$, which can preserve the distances between any two points with multiplicative error $\epsilon^{\prime}$.
For any point $x=(x_0,x_1,...,x_{n-1})$, suppose $f(x):=(f_i(x))_{i\in T}$. For each non-negative integer $j< n$, suppose $\gamma_j \in \{-1, +1\}$ is a uniform random
bit such that $\gamma$'s are independent. Let $f_i(x)=\frac{1}{\sqrt{T}}\sum_{j}\gamma_jx_j$.
By the proof of Johnson-Lindenstrauss lemma, for small $\epsilon^{\prime}$, $f$ can preserve the distances between any two points with multiplicative error $\epsilon^{\prime}$.
Furthermore, $f$ is a linear mapping. Since triangle $ZP_{XY1}P_{XY2}$ and triangle $XYZ$ are in the same plane,
the area of triangle $ZP_{XY1}P_{XY2}$ and triangle $XYZ$ under $f$ have the same multiplicative error.
We have proved that triangle $ZP_{XY1}P_{XY2}$ is preserved by multiplicative error, thus triangle $XYZ$ is also preserved by multiplicative error.
If we add at most another two points to all triangles that the $n$ points can construct, 
then all triangles fromed from the $n$ points are preserved. 
Since there are $n$ points, we can use the $n$ points to construct at most $\binom{n}{3}$ triangles.
Thus, the totally number of points is at most $n+2\binom{n}{3}$.

By Johnson-Lindenstrauss lemma, for any $0<\epsilon^{\prime\prime}<1$, suppose $U$ is a set of $n^{\prime\prime}$ points in Euclidean space $\mathbb{R}^{n^{\prime \prime}}$.
Let $T=\lceil \frac{12\ln2n^{\prime\prime 2}}{\epsilon^{\prime\prime 2}} \rceil$ where $n^{\prime\prime}$ is the number of points. 
Mapping $f: U\rightarrow \mathbb{R}^T$,
can ensure $(1-\epsilon^{\prime\prime})||x-y||^2\le||f(x)-f(y)||^2\le(1+\epsilon^{\prime\prime})||x-y||^2$.
%Since for small $\epsilon^{\prime\prime}$, $(1+\epsilon^{\prime\prime})^2=1+\Theta(\epsilon^{\prime\prime})$, $(1-\epsilon^{\prime\prime})^2=1-\Theta(\epsilon^{\prime\prime})$,
%it follows that the squared of the distances are preserved if and only if the distances themselves are.
\begin{align}
    \nonumber &(1-\epsilon^{\prime\prime})||x-y||^2\le||f(x)-f(y)||^2\le (1+\epsilon^{\prime\prime})||x-y||^2\\
    \nonumber &\implies \sqrt{1-\epsilon^{\prime\prime}}||x-y||\le||f(x)-f(y)||\le \sqrt{1+\epsilon^{\prime\prime}}||x-y||\\
    \nonumber &\implies (\sqrt{1-\epsilon^{\prime\prime}}-1)||x-y||\le ||f(x)-f(y)||-||x-y||\le (\sqrt{1+\epsilon^{\prime\prime}}-1)||x-y||\\
    \nonumber &\implies |||f(x)-f(y)||-||x-y|||\le \max( 1- \sqrt{1-\epsilon^{\prime\prime}},\sqrt{1+\epsilon^{\prime\prime}}-1 )||x-y||\\
    \nonumber &\implies |||f(x)-f(y)||-||x-y|||\le ( 1- \sqrt{1-\epsilon^{\prime\prime}})||x-y||
\end{align}
We have proved that 
if the distances between any two points is preserved by $\epsilon^{\prime}$, the area of any triangle is preserved by $1-\sqrt{-14\epsilon^{\prime 2}-4\epsilon^{\prime}+1}$.
Then, if the distances between any two points is preserved by $1- \sqrt{1-\epsilon^{\prime\prime}}$, $\epsilon^{\prime\prime} \in \left(0,1-\left(1-\frac{3\sqrt{2}-2}{14}\right)^2\right]$, 
the area of any triangle is preserved by $1-\sqrt{14\epsilon^{\prime\prime}+32(1-\epsilon^{\prime\prime})^{\frac{1}{2}}-31}\le 1-\sqrt{14\epsilon^{\prime\prime}+32(1-\epsilon^{\prime\prime})-31}=1-\sqrt{1-18\epsilon^{\prime\prime}}$.
After the above transformation, we should let $\epsilon^{\prime\prime} \in (0,\frac{1}{18}]$.
Since we have to guarantee  multiplicative error of the area of triangles with in $\epsilon$, 
we should let  $1-\sqrt{1-18\epsilon^{\prime\prime}}=\epsilon \implies \epsilon^{\prime\prime}=\frac{2\epsilon-\epsilon^2}{18}$ 
then we can ensure that $\epsilon^{\prime\prime} \in (0,\frac{1}{18}]$.
Then $T=\Bigg\lceil \frac{12\ln\left(2\cdot \left(n+2\binom{n}{3}\right)^2\right)  }{\epsilon^{\prime \prime 2}} \Bigg\rceil$, where $\epsilon^{\prime\prime}=\frac{2\epsilon-\epsilon^2}{18}$.