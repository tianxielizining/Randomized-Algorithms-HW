\section{Chernoff Bound: Measure Concentration}
\begin{theorem}[Basic Chernoff Bound]
    Suppose $Y$ is the sum of $n$ independent $\{0,1\}$-random variables $X_i$'s such that for each $i$, $Pr[X_i=1]=p$. 
    Let $\mu :=E[Y]=np$. Then, for $0<\epsilon<1$,
    \begin{align}
       \nonumber Pr[|Y-E[Y]|\le \epsilon E[Y]]\le 2e^{-\frac{1}{3}\epsilon^2np}
    \end{align}
\end{theorem}
\subsection{Homework}
\blue{1. \textbf{The Other Half of Chernoff.} Suppose $X_0, X_1, ..., X_{n-1}$ are independent $\{0,1\}$-random
variables, each having expectation $p$. Let $Y := \sum_{i}X_i$ and $\mu := E[Y]$. Using the method of
moment generating function, prove the following.
\begin{align}
    \nonumber \text{For all}~ \epsilon >0 \text{, }Pr[Y-\mu \ge\epsilon \mu] \le (\frac{e^{\epsilon}}{(1+\epsilon)^{1+\epsilon}})^{\mu}
\end{align}}
\begin{proof}
Set $t>0$, then we have:
\begin{align}
    \nonumber Pr[Y-\mu \ge \epsilon \mu] &=Pr[\sum_{i}X_i-\mu \ge \epsilon \mu]\\
    \nonumber &=Pr[t\sum_{i}X_i \ge t\mu(1+\epsilon)]\\
    \nonumber &=Pr[e^{t\sum_{i}X_i } \ge e^{t\mu(1+\epsilon)}]
\end{align}
By Markov's inequality, we have:
\begin{align}
    \nonumber Pr[e^{t\sum_{i}X_i } \ge e^{t\mu(1+\epsilon)}] \le \frac{E[e^{t\sum_{i}X_i }]}{e^{t\mu(1+\epsilon)}}
\end{align}
Since $X_i$'s are independent, we have:
\begin{align}
    \nonumber E[e^{t\sum_{i}X_i }]=\prod_{i}E[e^{tX_i}]
\end{align}
By definition, for each $Pr[X_i=1]=p$, $Pr[X_i=0]=1-p$, and the fact that for all real numbers $x$, $1+x\le e^x$, we have:
\begin{align}
    \nonumber E[e^{tX_i}]=e^t\cdot p+e^0 \cdot (1-p)=1+p(e^t-1) \le e^{p(e^t-1)}
\end{align}
Then,
\begin{align}
    \nonumber \prod_{i}E[e^{tX_i}]\le e^{np(e^t-1)}
\end{align}
Thus, we have:
\begin{align}
    \nonumber Pr[Y-\mu \ge \epsilon \mu] \le \frac{e^{np(e^t-1)}}{e^{t\mu(1+\epsilon)}}=e^{np(e^t-1)-\mu t(1+\epsilon)}=e^{\mu(e^t-(1+\epsilon)t-1)}
\end{align}
Set $g(t)=\mu(e^t-(1+\epsilon)t-1)$. We aim to find the value $t$ that minimizes the function $g(t)$.
We have $g'(t)=\mu(e^t-(1+\epsilon))$ and $g''(t)=\mu e^t >0$. It follows that function $g(t)$ attains its minimum when $g'(t) = 0$.
Check that $t>0$, so we can set $t=\ln{(1+\epsilon)}$.
And $g(\ln{(1+\epsilon)})=\mu(1+\epsilon-(1+\epsilon)\ln{(1+\epsilon)}-1)=\mu(\epsilon-(1+\epsilon)\ln(1+\epsilon))$.
Thus,
\begin{align}
    \nonumber Pr[Y-\mu \ge \epsilon \mu] &\le e^{\mu(\epsilon-(1+\epsilon)\ln(1+\epsilon))}\\
    \nonumber &=(\frac{e^\epsilon}{e^{(1+\epsilon)\ln(1+\epsilon)}})^{\mu}\\
    \nonumber &=(\frac{e^\epsilon}{(1+\epsilon)^{1+\epsilon}})^{\mu}
\end{align}
\end{proof}
\noindent\blue{2. \textbf{$n$ Balls into $n$ Bins (Revisited).} 
Using the Chernoff Bound from the previous question, we can obtain a better bound for the balls and bins problem. 
Suppose $n$ balls are thrown independently and uniformly at random into $n$ bins. 
Let $Y_1$ be the number of balls in the first bin. Assume that $n$ is large enough, say $n\ge 100$.\\
(a) Find a number $N$ in terms of $n$ such that $Pr[Y_1\ge N]\le \frac{1}{n^2}$. 
Please give the exact form and do not use big $O$ notation for this part of the question.
}\\
\begin{claim}\label{1}
    For all $n\ge e^2$, $\ln \ln n -\ln \ln \ln n \ge \frac{\ln \ln n}{2}$.
\end{claim}
\begin{proof}
    Set: $f(x)=lnx$, $x\ge \ln 2$ and $x_1=\ln n$, $x_2=\ln\ln n$, then we have to proof $\ln x_1 \ge 2 \ln x_2$.
    It suffices to prove that $x_1 \ge (x_2)^2$.
    Set $g(n)=x_1-(x_2)^2=\ln n- (\ln\ln n)^2$, $n \ge e^2$. Then $g'(n)=\frac{1}{n}(1-\frac{2\ln \ln n}{\ln n})$.
    Set: $u(n)=\ln n - 2\ln \ln n$, $n \ge e^2$. Then $u'(n)=\frac{1}{x} \cdot (1-\frac{2}{\ln x})$.
    Since $n\ge e^2$, we have $u'(n)\ge 0$. It follows that function $u(n)$ attains its minimum when $n=e^2$ and $u(e^2)=2-2\ln 2>0$.
    Then we have $u(n)>0$ for all $n \ge e^2$. Then, it implies that $\frac{2\ln \ln n}{\ln n}<1$ and $1-\frac{2\ln \ln n}{\ln n}>0$.
    Then  $g'(n)> 0$, which implies that function $g(n)$ attains its minimum when $n=e^2$ and $g(e^2)=2-(\ln 2)^2>0$.
    Then $x_1 \ge (x_2)^2$ when $n\ge e^2$.

\end{proof}
\noindent By definition, $E[Y_1]=\frac{1}{n}\cdot n =1$.
By Chernoff Bound, we have,
\begin{align}
    \nonumber Pr[Y_1\ge 1+\epsilon] \le \frac{e^\epsilon}{(1+\epsilon)^{1+\epsilon}}
\end{align}
Since we have to satisfy $Pr[Y_1\ge N] \le \frac{1}{n^2}$, we can set $N:=1+\epsilon$ and have,
\begin{align}
    \nonumber \frac{e^\epsilon}{(1+\epsilon)^{1+\epsilon}}&\le \frac{1}{n^2}\\
    \nonumber \frac{e^{N-1}}{N^N}&\le \frac{1}{n^2}\\
    \nonumber \frac{N^N}{e^{N-1}}&\ge n^2\\
    \nonumber  \ln (\frac{N^N}{e^{N-1}})&\ge \ln (n^2)\\
    \nonumber N \ln N - (N-1) &\ge 2\ln n \\
    \nonumber N(\ln N -1) & \ge 2\ln n +1
\end{align}
Set $N:=\frac{\lambda \ln n}{\ln \ln n}$, we need to satisfy:
\begin{align}
    \nonumber \frac{\lambda \ln n}{\ln \ln n}(\ln (\frac{\lambda \ln n}{\ln \ln n})-1) \ge 2\ln n +1\\
    \nonumber \frac{\lambda \ln n}{\ln \ln n}(\ln \lambda +\ln\ln n -\ln \ln \ln n -1) \ge 2\ln n +1
\end{align}

%Since $\ln \ln n -\ln \ln \ln n \ge \frac{\ln \ln n}{2}$, 
By Claim \ref{1},
we have,
\begin{align}
    \nonumber \frac{\lambda \ln n}{\ln \ln n}(\ln \lambda +\ln n -\ln \ln \ln n -1) &\ge \frac{\lambda \ln n}{\ln \ln n}(\ln \lambda -1 +\frac{\ln \ln n}{2})\\
    \nonumber &=\frac{\lambda \ln n}{\ln \ln n}\cdot \frac{\ln\ln n}{2}+\frac{\lambda \ln n}{\ln \ln n}\cdot(\ln\lambda-1)\\
    \nonumber &=\frac{\lambda \ln n}{2} +\frac{\lambda \ln n}{\ln \ln n}\cdot(\ln\lambda-1)
\end{align}
Set $\lambda \ge e$, then, $\ln\lambda-1 \ge 0$. Thus,
\begin{align}
    \nonumber \frac{\lambda \ln n}{\ln \ln n}(\ln \lambda +\ln n -\ln \ln \ln n -1) &\ge \frac{\lambda \ln n}{2}
\end{align}
Then we have to satisfy:
$\frac{\lambda \ln n}{2} \ge 2\ln n +1$ and $\lambda \ge e$ at the same time. So we have:
\begin{align}
    \nonumber \lambda \ge \frac{4\ln n +2}{\ln n}
\end{align}
Since $N=\frac{\lambda \ln n}{\ln \ln n}$, we have:
\begin{align}
    \nonumber N&\ge \frac{4\ln n +2}{\ln n}\cdot \frac{\ln n}{\ln \ln n}\\
    \nonumber &= \frac{4\ln n +2}{\ln \ln n}
\end{align}
\noindent\blue{(b) Show that with probability at least $1-\frac{1}{n}$, no bin contains more than $\Theta(\frac{\log n}{\log\log n})$ balls.}\\
Let $B_1$ be the event that there exists a bin that with more than $\frac{4\ln n +2}{\ln \ln n}$ balls.
By union bound, we have $Pr[B_1]\le\frac{1}{n^2}\cdot n=\frac{1}{n}$.
Let $B_2$ be the event that no bin contains more than $\frac{4\ln n +2}{\ln \ln n}$ balls.
Then, $Pr[B_2]\ge 1-\frac{1}{n}$.\\
Since $\frac{4\ln n +2}{\ln \ln n}=\Theta(\frac{\log n}{\log\log n})$, we have that with probability at least $1-\frac{1}{n}$, no bin contains more than $\Theta(\frac{\log n}{\log\log n})$ balls.