\documentclass[a4paper,11pt]{article}
\usepackage[utf8]{inputenc}
\usepackage{xeCJK}%支持中文
\usepackage{colortbl}%彩色表格
\usepackage{bm}%处理数学公式
\usepackage{amsmath}
\usepackage{amssymb}
\usepackage{amsthm}
\usepackage[ruled,vlined]{algorithm2e}     
\usepackage[noend]{algpseudocode}%算法伪代码
\usepackage{cases}
\usepackage{wrapfig}%图表和文字混排宏包
\usepackage{mathrsfs}%花体
%\usepackage{balance}%最后一页平衡balance the last two columns in twocolumn mode
%\usepackage[linesnumbered,boxed,ruled,vlined]{algorithm2e}%算法排版样式
\usepackage{graphicx}%加入图片
%\usepackage{afterpage}
\usepackage{geometry}%可以自定义页面设置
\geometry{margin=1in}
\usepackage{tabularx}% 用于灵活地控制表格的生成
\usepackage{hyperref}% 文献引用的宏包
\hypersetup{colorlinks=true,allcolors=blue}
\usepackage{xcolor}%改变字体颜色
\usepackage{caption}%图表标题
\usepackage{subcaption}%子标题
\usepackage{enumerate}
\usepackage{bm}
\newcommand{\lzn}[1]{\textcolor{purple}{#1}}
\newcommand{\tann}[1]{\textcolor{red}{#1}}
\newcommand{\para}[1]{{\vspace{3pt} \bf \noindent #1}}
\newcommand{\blue}[1]{\textcolor{blue}{#1}}
\theoremstyle{plain}
\newtheorem{theorem}{Theorem}[section]
\newtheorem{lemma}[theorem]{Lemma}
\newtheorem{claim}[theorem]{Claim}
\newtheorem{corollary}[theorem]{Corollary}
\newtheorem{question}[theorem]{Question}
\newtheorem{proposition}[theorem]{Proposition}
\newtheorem{fact}[theorem]{Fact}
\theoremstyle{definition}
\newtheorem{definition}[theorem]{Definition}
\theoremstyle{remark}
\newtheorem{remark}[theorem]{Remark}
\newcommand{\rmnum}[1]{\romannumeral #1}
\newcommand{\Rmnum}[1]{\expandafter\@slowromancap\romannumeral #1@}

\begin{document}
%\section{Preliminaries}
\blue{1. Prove the union bound: Suppose $\{A_i\}_{i=1}^n$ is a collection of events. Then, $\Pr[\cup_{i=1}^{n}A_i]\le\sum_{i=1}^{n}\Pr[A_i]$.}
\begin{proof}
Let $t \in [n]$.
Prove the above by induction on $t$.
Certainly the inequality holds for the base case where $t=1$, since $\Pr[A_1]=\Pr[A_1]$.
Suppose the inequality holds for $n=t-1$ and if I can prove that the inequality holds for $n=t$ then finish the proof.
\begin{align}
    \nonumber \Pr[\cup_{i=1}^{n}A_i]&=\Pr[\cup_{i=1}^{n-1}A_i\cup A_n]\\
    \nonumber&= \Pr[\cup_{i=1}^{n-1}A_i]+\Pr[A_n]-\Pr[(\cup_{i=1}^{n-1}A_i)\cap A_n]\\
    \nonumber&\le \sum_{i=1}^{n-1}\Pr[A_i]+\Pr[A_n]\\
    \nonumber&=\sum_{i=1}^{n}\Pr[A_i]
\end{align}
\end{proof}
\noindent\blue{2. Prove the linearity of expectation: Suppose for each $1 \le i \le n$, $X_i$ is a random variable and $a_i$ is a real number. Then, $E[\sum_{i=1}^na_iX_i]=\sum_{i=1}^{n}a_iE[X_i]$.}
\cite{linearity}
\begin{definition}[\cite{Rajeev1995Randomized}]\label{def1}
The expectation of a random variable $X$ with density function $p$ is defined as $E[X] = \sum_x xp(x)$, where the summation is over the range of $X$.
\end{definition}
\begin{lemma}\label{lemma2}
For any random variable $X$ and $Y$, $E[X+Y]=E[X]+E[Y]$.
\end{lemma}
\begin{proof}
Prove by definition \ref{def1}.
$\sum_x$ and $\sum_y$ means the summation is over the range of $X$ and $Y$.
\begin{align}
    \nonumber E[X+Y]&=\sum_x\sum_y(x+y)\Pr[(X=x)\cap(Y=y)]\\
    \nonumber&=\sum_x\sum_yx\Pr[(X=x)\cap(Y=y)]+y\Pr[(X=x)\cap(Y=y)]\\
    \nonumber&=\sum_xx\sum_y\Pr[(X=x)\cap(Y=y)]+\sum_yy\sum_x\Pr[(X=x)\cap(Y=y)]\\
    \nonumber&=\sum_xx\Pr[X=x]+\sum_yy\Pr[Y=y]\\
    \nonumber&=\sum_xxp(x)+\sum_yyp(y)\\
    \nonumber&=E[X]+E[Y]
\end{align}
\end{proof}
\begin{lemma}\label{lemma3}
For any random variable $X$ and any constant $a \in \mathbb{R}$, $E[aX]=aE[X]$.
\end{lemma}
\begin{proof}
Prove by definition \ref{def1}.
$\sum_x$ means the summation is over the range of $X$.
\begin{align}
    \nonumber E[aX]&=\sum_xaxp(x) %~~~~\text{the summation is over the range of }X  
    \\
    \nonumber &=a\sum_xxp(x)% ~~~~\text{the summation is over the range of }X
    \\
    \nonumber &=aE[X]
\end{align}
\end{proof}
By combining Claim \ref{lemma2} and \ref{lemma3}, we have:
\begin{align}
    \nonumber E[\sum_{i=1}^na_iX_i]&=\sum_{i=1}^nE[a_iX_i]\\
    \nonumber&=\sum_{i=1}^{n}a_iE[X_i]
\end{align}
\blue{
3. Suppose $X$, $Y: \Omega \rightarrow \mathbb{Z}$ are independent random variables taking integer values. In this question, we shall prove the basic result $E[XY]=E[X]E[Y]$.}
\begin{proof}
Prove by definition \ref{def1}.
$\sum_x$ and $\sum_y$ means the summation is over the range of $X$ and $Y$.
\begin{align}
    \nonumber E[XY]&=\sum_x\sum_yxy\Pr[(X=x)\cap(Y=y)]\\
    \nonumber &=\sum_x\sum_yxy\Pr[X=x]\Pr[Y=y]\\
    \nonumber&=\sum_xx\Pr[X=x]\sum_yy\Pr[Y=y]\\
    \nonumber&=\sum_xxp(x)\sum_yyp(y)\\
    \nonumber&=E[X]E[Y]
\end{align}
\end{proof}
\noindent\blue{4. Let $\phi$ be a $3$-CNF formula with $m$ clauses and $n$ variables. \\
(a) Let $Y$ be the random variable denoting the number of unsatisfied clauses. Compute $E[Y]$.\\
(b) Find a suitable upper bound for $\Pr[Y>\frac{3m}{16}]$, and conclude that the randomized algorithm finds an assignment satisfying at least $\frac{13m}{16}$ clauses with probability at least $\frac{1}{3}$.}\\
\textbf{Problem description:} There are $n$ Boolean variables. $\phi(x_1,x_2,...,x_n)$ is denoted as $C_1 \vee C_2 \vee...\vee C_m$. Each clause $C_i$ is a disjunction of $3$ literals from $3$ different variables. A literal is either a variable or its negation. A clause is satisfied if at least one of its $3$
literals evaluates to TRUE. The goal is to find an assignment of the variables so that as many clauses as possible are satisfied.\\
\textbf{Randomized algorithm:} Independently for each variable, assign its value to be TRUE or FALSE, each with probability $\frac{1}{2}$.\\
(a)
Define $Y_i$ as a random variable which is related to clause $C_i$. $Y_i$ takes value 1 if the clause $C_i$ is unsatisfied and 0 otherwise. Then the number of unsatisfied clauses $Y=\sum_{i=1}^{m}Y_i$. For each clause $C_i$, there are $3$ independent Boolean variables. And the value of each Boolean variable is independently assigned TRUE or FALSE, each with probability $\frac{1}{2}$. Since for each clause $C_i$ is unsatisfied only if all the three literals evaluate to FALSE, for each $Y_i$, takes value $1$ with probability $\frac{1}{8}$ and takes value $0$ with probability $\frac{7}{8}$. By Lemma \ref{lemma2}:
\begin{align}
    \nonumber E[Y]&=E[\sum_{i=1}^{m}Y_i]\\
    \nonumber&=\sum_{i=1}^mE[Y_i]\\
    \nonumber&=m(1\cdot \frac{1}{8}+0 \cdot \frac{7}{8})\\
    \nonumber&=\frac{m}{8}
\end{align}
(b)By Markov's inequality, for all $\alpha> 0$ we have
\begin{align}
   \nonumber \Pr[Y\ge \alpha]\le \frac{E[Y]}{\alpha}
\end{align}
If set $\alpha=\frac{3m}{16}$,we have
\begin{align}
    \nonumber \Pr[Y\ge \frac{3m}{16}]\le \frac{m}{8}\cdot \frac{16}{3m}=\frac{2}{3}
\end{align}
Thus, 
\begin{align}
    \nonumber \Pr[Y>\frac{3m}{16}] \le \Pr[Y\ge \frac{3m}{16}]\le \frac{2}{3} ~\implies~\Pr[Y\le \frac{3m}{16}]\ge \frac{1}{3}
\end{align}
Thus, the randomized algorithm finds an assignment satisfying at least $\frac{13m}{16}$ clauses with probability at least $\frac{1}{3}$.\\
\blue{5. Let $G=(V, E)$ be a graph. Recall the randomized algorithm mentioned in class for finding a cut $C\subset V$ for the graph $G$, namely, a point $v \in V$ is included in $C$ independently with probability $\frac{1}{2}$. Assume that to make this decision takes $1$ independent random bit for each point.\\
Let $E(C):=\{\{u,v\}\in E: u\in C , v\in V\setminus C\}$ be the edges in the cut. It is shown that $E[|E(C)|]=\frac{|E|}{2}$. The goal of this question is to design another randomized algorithm with better guarantees. Let $0<\epsilon<1$ and $0< \delta < 1$. We shall design a randomized algorithm that, with failure probability at most $\delta$, returns a cut $C$ such that $|E(C)| \ge (\frac{1-\epsilon}{2})\cdot |E|$.\\
(a) Given an upper bound on the failure probability that the above randomized procedure returns a cut such that the number of edges $|E(C)|$ is less than $(\frac{1-\epsilon}{2})\cdot |E|$.\\
(b) Show that by repeating the above randomized procedure, it is possible to obtain a better randomized algorithm with failure probability at most $\delta$. Compute the number of independent random bits used by your algorithm.}
\cite{maxcut}\cite{reverse}\\
(a)
\begin{align}
  \nonumber \Pr[|E(C)|\le(\frac{1-\epsilon}{2})|E|]=\Pr[|E|-|E(C)|\ge |E|-\frac{1-\epsilon}{2}|E|]
\end{align}
By Markov's inequality, we have:
\begin{align}
  \nonumber \Pr[|E|-|E(C)|\ge |E|-\frac{1-\epsilon}{2}]\le \frac{E[|E|-|E(C)|]}{|E|-\frac{1-\epsilon}{2}|E|}
\end{align}
By Lemma \ref{lemma2}, we have:
\begin{align}
  \nonumber E[|E|-|E(C)|]=|E|-E[|E(C)|]=|E|-\frac{|E|}{2}=\frac{|E|}{2}
\end{align}
Thus,
\begin{align}
\nonumber \Pr[|E(C)|\le(\frac{1-\epsilon}{2})|E|]\le \frac{\frac{|E|}{2}}{|E|-\frac{1-\epsilon}{2}|E|}=\frac{1}{1+\epsilon}
\end{align}
(b)The random algorithm is described as follows.\\
\begin{enumerate}[(i)]
    \item For each vertex $v\in V$, independently assign it a number 0 or 1, each with probability $\frac{1}{2}$. Then we can get a set of vertices $C$ which consists of the vertices with number 1.
    \item Independently repeat step (i) for $k$ times, here, \blue{$k=-\frac{2}{\epsilon}\ln \delta$}, which is the answer to the question. We can get $k$ cuts of graph $G$, named $C_1, C_2,...,C_k$.
    \item For each cut $C_i$, compute $|E(C_i)|$, which is the number of edges in cut $C_i$. Then pick $C_j$ from all $C_i$s with the maximum $|E(C_i)|$. And then output $C_j$ as the final result. 
\end{enumerate}
\textbf{Analysis:}
By (a), for any $C_i$ we have
\begin{align}
    \nonumber \Pr[|E(C_i)|\le (\frac{1-\epsilon}{2})\cdot |E|]\le \frac{1}{1+\epsilon}
\end{align}
Since $C_j$ is with the maximum $|E(C_i)|$ among all $C_j$s, we have
\begin{align}
    \nonumber \Pr[|E(C_j)|\le (\frac{1-\epsilon}{2})\cdot |E|]&=\prod_{i=1}^k \Pr[|E(C_i)|\le (\frac{1-\epsilon}{2})\cdot |E|]\\
    \nonumber &\le (\frac{1}{1+\epsilon})^k
\end{align}
Since for $0< \epsilon <1$, $1+\epsilon \ge \exp({\frac{\epsilon}{2}})$, we have
\begin{align}
    \nonumber \Pr[|E(C_j)|\le (\frac{1-\epsilon}{2})\cdot |E|]&\le \exp({-\frac{\epsilon k}{2}})\\
    \nonumber&=\exp({-\frac{\epsilon}{2}(-\frac{2}{\epsilon}\ln \delta)})\\
    \nonumber&=\delta
\end{align}
\blue{
6.(a) Suppose $X$ is a discrete random variable that takes only a countable number of non-negative values. Prove that $E[X]=\int_{0}^{\infty}\Pr[X\ge t]\, dt$. (If you do not like to deal with infinite support, you may assume $X$ only takes a finite number of values.)\\
(b) Assume that for all non-negative random variables $X$, $E[X]=\int_{0}^{\infty}\Pr[X\ge t]\, dt$. Derive a similar formula for $E[X]$ when $X$ is any real-valued random variable.}\\
(a) 
First assume discrete random variable $X$ only takes a finite number $n$ of non-negative values, $x_1, x_2,...,x_n$ with probability $p_{x1},p_{x2},...,p_{xn}$. Note that $\sum_{i=1}^{n}p_{xi}=1$. 
By definition, $E[X]=\sum_{i=1}^{n}x_ip_{xi}$.
For simplicity, assume $0\le x_1< x_2 < ... < x_n$.
Then we can get a piecewise function as follows.
\begin{equation}
\nonumber \Pr[X\ge t]=\left\{
\begin{array}{rcl}
1 & & {0\le t \le x_1}\\
1-p_{x1} & & {x_1 < t \le x_2}\\
1-\sum_{i=1}^{2}p_{xi} & & {x_2 < t \le x_3}\\
1-\sum_{i=1}^{3}p_{xi} & & {x_3 < t \le x_4}\\
... & & ...\\
1-\sum_{i=1}^{n-1}p_{xi} & & {x_{n-1} < t \le x_n}\\
0 & & {x_n \le t}
\end{array} \right.
\end{equation}
Thus we have:
\begin{align}
  \nonumber\int_{0}^{\infty}\Pr[X\ge t]\, dt &= (x_1-0)\cdot 1+(x_2-x_1) (1-p_{x1})+(x_3-x_2) (1-\sum_{i=1}^{2}p_{xi})+\\
  \nonumber&(x_4-x_3) (1-\sum_{i=1}^{3}p_{xi})+...+(x_n-x_{n-1})(1-\sum_{i=1}^{n-1}p_{xi})+0\cdot \infty\\
  \nonumber&=x_1(1-1+p_{x1})+x_2(1-p_{x1}-(1-\sum_{i=1}^2p_{x_i}))+x_3(1-\sum_{i=1}^2p_{x_i}-(1-\sum_{i=1}^3p_{x_i}))\\
  \nonumber&~~~~+...+x_n(1-\sum_{i=1}^{n-1}p_{x_i}-(1-\sum_{i=1}^np_{x_i}))\\
  \nonumber&=x_1p_{x1}+x_2p_{x2}+x_3p_{x3}+...+x_np_{xn}\\
  \nonumber&=E[X]
\end{align}
Then consider the situation when discrete random variable $X$ takes infinite number of non-negative values (but the number of values is still countable).
Assume $X$ can take values $0\le x_1< x_2 < x_3 < ...$ with probability $p_{x1},p_{x2},p_{x3},...$ and $\int_{0}^{\infty}\Pr[X=t]\, dt=1$. By definition, $E[X]=\sum_{k=0}^{\infty}k\Pr[X=k]$ and $\Pr[X\ge t]=\sum_{k=t}^{\infty}\Pr[X=k]$. Thus, we have:
\begin{align}
  \nonumber\int_{0}^{\infty}\Pr[X\ge t]\, dt &=\int_{0}^{\infty}\sum_{k=t}^{\infty}\Pr[X=k] \, dt\\
  \nonumber&=\int_{0}^{x_1}\sum_{k=x_1}^{\infty}\Pr[X=k] \, dt + \int_{x_1}^{x_2}\sum_{k=x_2}^{\infty}\Pr[X=k] \, dt + \int_{x_2}^{x_3}\sum_{k=x_3}^{\infty}\Pr[X=k] \, dt +...\\
  \nonumber&=(x_1-0)\sum_{k=x_1}^{\infty}\Pr[X=k]+(x_2-x_1)\sum_{k=x_2}^{\infty}\Pr[X=k]+(x_3-x_2)\sum_{k=x_3}^{\infty}\Pr[X=k]+...\\
  \nonumber&=x_1(\sum_{k=x_1}^{\infty}\Pr[X=k]-\sum_{k=x_2}^{\infty}\Pr[X=k])+x_2(\sum_{k=x_2}^{\infty}\Pr[X=k]-\sum_{k=x_3}^{\infty}\Pr[X=k])+\\
  \nonumber&~~~~ x_3(\sum_{k=x_3}^{\infty}\Pr[X=k]-\sum_{k=x_2}^{\infty}\Pr[X=k])+...\\
  \nonumber&=x_1\Pr[X=x_1]+x_2\Pr[X=x_2]+x_3\Pr[X=x_3]+...\\
 % \nonumber&=x_1p_{x1}+x_2p_{x2}+x_3p_{x3}+...\\
  \nonumber &=\sum_{k=0}^{\infty}k\Pr[X=k]\\
  \nonumber &=E[X]
\end{align}
\\
(b)
Let $X$ be a continuous real-valued random variable. The probability density function of $X$ is $f(x)$. By definition, $\int_{\mathbb{R}}f(x)\, dx=1$, $E[X]=\int_{\mathbb{R}}xf(x)\, dx$. $E[X]$ can also be written as
$E[X]=-\int_{-\infty}^{0}\Pr[X\le t]\, dt+\int_{0}^{\infty}\Pr[X\ge t]\, dt$
\begin{proof}
\begin{align}
    \nonumber \int_{0}^{\infty}\Pr[X\ge t]\, dt &=\int_{0}^{\infty}
    \int_{t}^{\infty}f(k)\, dk
    \, dt\\
    \nonumber&=\int_{0}^{\infty}
    kf(k)
    \, dk\\
    \nonumber&=\int_{0}^{\infty}
    xf(x)
    \, dx
\end{align}
\begin{align}
    \nonumber -\int_{-\infty}^{0}\Pr[X\le t]\, dt &=-\int_{-\infty}^{0}
    \int_{-\infty}^{t}f(k)\, dk
    \, dt\\
    \nonumber&=-\int_{0}^{-\infty}
   \int_{t}^{-\infty}f(k)\, dk
    \, dt\\
    \nonumber&=-\int_{0}^{-\infty}
    kf(k)
    \, dk\\
    \nonumber&=\int_{-\infty}^{0}
    xf(x)
    \, dx
\end{align}
\begin{align}
    \nonumber E[X]&=\int_{\mathbb{R}}xf(x)\, dx\\
    \nonumber &=\int_{-\infty}^{0}xf(x)\, dx+\int_{0}^{\infty}xf(x)\, dx\\
    \nonumber &=-\int_{-\infty}^{0}\Pr[X\le t]\, dt + \int_{0}^{\infty}\Pr[X\ge t]\, dt
\end{align}
\end{proof}
%\section{A Proof using Stirling's Formula}
\begin{lemma}[\textbf{Stirling's formula}\cite{stirling}]\label{Stirling}
 For $n=1,2,3,...$
\begin{align}
    \nonumber \sqrt{2\pi}n^{n+\frac{1}{2}}e^{-n}e^{\frac{1}{12n+1}}< n! < \sqrt{2\pi}n^{n+\frac{1}{2}}e^{-n}e^{\frac{1}{12n}} 
\end{align}
\end{lemma}
\begin{definition}[\textbf{Euler's number}\cite{e}]\label{e}
The number $e$ is a mathematical constant approximately equal to $2.71828$. It is the limit of $(1+\frac{1}{n})^n$ as $n$ approaches infinity. The number $e$ is the unique real number such that
\begin{align}
    \nonumber (1+\frac{1}{x})^{x} < e <(1+\frac{1}{x})^{x+1}
\end{align}
\end{definition}
\begin{lemma}\label{gp}
Given a geometric progression with first term $a> 0$ and common ratio $r \ge 2$, for every $n$, the sum of the first $n$ terms is less than the ($n+1$)-th term.
%\begin{align}
 %  \nonumber \sum_{k=1}^{n}ar^{k-1} textless ar^{n}
%\end{align}
\end{lemma}
\begin{proof}
%Suppose a geometric progression with first term $a$ and common ratio $r$ as follows:
%\begin{align}
 %   \nonumber a, ar, ar^2, ar^3, ar^4, ...
%\end{align}
%If $r \ge 2$ and $a\textgreater 0$, then the sum of first $n-1$ terms is less than the value of the $n$-th term.
%\begin{align}
 %   \nonumber \sum_{k=1}^{n-1}ar^{k-1} %\textless ar^{n-1}
%\end{align}
For every $n$, the sum of the first $n$ terms is $\sum_{k=1}^{n}ar^{k-1}$ and the ($n+1$)-th term is $ar^{n}$.
\begin{align}
    \nonumber \sum_{k=1}^{n}ar^{k-1}&=\frac{a(r^{n}-1)}{r-1}\\
    \nonumber &=\frac{ar^{n}-a}{r-1}
\end{align}
\begin{align}
    \nonumber &r\ge 2 \implies r-1 \ge 1 \implies \frac{1}{r-1} \le 1
\end{align}
Thus, 
\begin{align}
    \nonumber \sum_{k=1}^{n}ar^{k-1}\le ar^{n}-a < ar^{n}
\end{align} 
\end{proof}


\begin{claim}
$\sum_{i=3}^{n}\tbinom{n}{i}p^i<(np)^l$
\end{claim}
\begin{proof}
Consider $\tbinom{n}{i}$, where $i\in\{3,4,5,...,l\}$. By Lemma \ref{Stirling}, we have:
\begin{align}
   \nonumber\tbinom{n}{i}&=\frac{n!}{i!(n-i)!}\\
    \nonumber&< \frac{\sqrt{2\pi}n^{n+\frac{1}{2}}e^{-n}e^{\frac{1}{12n}}}{\sqrt{2\pi}i^{i+\frac{1}{2}}e^{-i}e^{\frac{1}{12i+1}}\sqrt{2\pi}(n-i)^{(n-i)+\frac{1}{2}}e^{-(n-i)}e^{\frac{1}{12(n-i)+1}}}\\
    \nonumber &=\frac{1}{\sqrt{2\pi}}\cdot e^{\frac{1}{12n}-\frac{1}{12i+1}}\cdot e^{-\frac{1}{12(n-i)-1}} \cdot n^{n+\frac{1}{2}} \cdot \frac{1}{i^{i+\frac{1}{2}}} \cdot \frac{1}{(n-i)^{n-i+\frac{1}{2}}}\\
    &=\frac{1}{\sqrt{2\pi}}\cdot e^{\frac{1}{12n}-\frac{1}{12i+1}}\cdot e^{-\frac{1}{12(n-i)-1}} \cdot \frac{n^i}{i^{i+\frac{1}{2}}} \cdot (\frac{n}{n-i})^{n-i+\frac{1}{2}}\label{a}
\end{align}
Since $n\ge 2^{l+2}$ and $l\ge i$, we have $12n > 12i+1$, which implies $\frac{1}{12n} < \frac{1}{12i+1}$. Then
\begin{align}
     e^{\frac{1}{12n}-\frac{1}{12i+1}} < 1 \label{b}
\end{align}
By Definition \ref{e}, bound $(\frac{n}{n-i})^{n-i+\frac{1}{2}}$ as follows.
\begin{align}
    \nonumber (\frac{n}{n-i})^{n-i+\frac{1}{2}}
    &= (\frac{n-i+i}{n-i})^{n-i+\frac{1}{2}}\\
    \nonumber&= (1+\frac{i}{n-i})^{n-i+\frac{1}{2}}\\
    \nonumber&= (1+\frac{1}{\frac{n-i}{i}})^{n-i+\frac{1}{2}}\\
    \nonumber&= ((1+\frac{1}{\frac{n-i}{i}})^{\frac{n-i}{i}})^{i+\frac{i}{2(n-i)}}\\
    \nonumber &< e^{i+\frac{i}{2(n-i)}}
\end{align}
Since $\tbinom{n}{i}=\tbinom{n}{n-i}$, 
we can assume $i\le \frac{n}{2}$. 
%the assume that $i\le \frac{n}{2}$ is reasonable. 
In order to bound $(\frac{n}{n-i})^{n-i+\frac{1}{2}}$, we have to bound $\frac{i}{2(n-i)}$
\begin{align}
   \nonumber i\le \frac{n}{2} 
   \implies n \ge 2i 
   %\implies n-i \ge i 
   \implies 2(n-i) \ge 2i \implies \frac{1}{2(n-i)} \le \frac{1}{2i} \implies  \frac{i}{2(n-i)} \le \frac{i}{2i} =\frac{1}{2}
\end{align}
Thus, 
\begin{align}
    (\frac{n}{n-i})^{n-i+\frac{1}{2}} <e^{i+\frac{1}{2}}\label{c}
\end{align}
By inequation \ref{a}, \ref{b} and \ref{c} we have:
\begin{align}
\nonumber \tbinom{n}{i} &<\frac{1}{\sqrt{2\pi}}\cdot 1\cdot e^{-\frac{1}{12(n-i)-1}} \cdot \frac{n^i}{i^{i+\frac{1}{2}}} \cdot e^{i+\frac{1}{2}}\\
\nonumber &= \frac{1}{\sqrt{2\pi}}\cdot e^{i+\frac{1}{2}-\frac{1}{12(n-i)-1}} \cdot \frac{n^i}{i^{i+\frac{1}{2}}}\\
\nonumber &< \frac{1}{\sqrt{2\pi}}\cdot e^{i+\frac{1}{2}} \cdot \frac{n^i}{i^{i+\frac{1}{2}}}\\
\nonumber &= \frac{1}{\sqrt{2\pi}}\cdot (\frac{e}{i})^{i+\frac{1}{2}}\cdot n^i
\end{align}
Since $i\ge 3$, we have $\frac{e}{i}< 1$. Thus,
\begin{align}
    \nonumber \tbinom{n}{i} < \frac{1}{\sqrt{2\pi}}\cdot n^i
\end{align}
Then, we have
\begin{align}
    \nonumber E[Y] &= \sum_{i = 3}^{l}{\binom{n}{i} p^i} 
    \\
    \nonumber &<\sum_{i=1}^{l}\frac{1}{\sqrt{2\pi}}n^ip^i\\
    \nonumber &= \frac{1}{\sqrt{2\pi}}\sum_{i=3}^{l}(np)^i
\end{align}
Since $p \ge \frac{2}{n}$, we have $np\ge 2$.
By Lemma \ref{gp}, we have:
\begin{align}
    \nonumber E[Y]& <   \frac{1}{\sqrt{2\pi}}\sum_{i=3}^{l}(np)^i \\
    \nonumber & < \frac{1}{\sqrt{2\pi}}\sum_{i=1}^{l}(np)^i \\
    \nonumber & < \frac{2}{\sqrt{2\pi}}(np)^l\\
    \nonumber & <(np)^l
\end{align}


\end{proof}
\section{Graphs with No Short Cycles}
\begin{definition}
An undirected graph $G = (V,E)$ contains a cycle of length $l$ if there are $l$ vertices $v_1,v_2,...,v_l$ such that all $l$ edges $\{v_1,v_2\},\{v_2,v_3\},...\{v_{l−1},v_l\},\{v_l,v_1\} \in E$ are present. The minimum length of a cycle is $3$.
\end{definition}
\noindent\textbf{Question.} Suppose a graph has no cycles of length $l$ or less. What is the maximum number of edges that it can have?

We will give a randomized algorithm and a deterministic algorithm in the following subsections. Both algorithms guarantee the number of edges.
\subsection{Preliminaries}
%The following lemmas and definition are useful for analysis.

\begin{lemma}[\textbf{linearity of expectation}]\label{linearity}
Suppose for each $1 \le i \le n$, $X_i$ is a random variable and $a_i$ is a real number. Then, $E[\sum_{i=1}^na_iX_i]=\sum_{i=1}^{n}a_iE[X_i]$.
\end{lemma}
\begin{lemma}\label{gp}
Given a geometric progression with first term $a \textgreater 0$ and common ratio $r \ge 2$, for every $n$, the sum of the first $n$ terms is less than the ($n+1$)-th term.
%\begin{align}
 %  \nonumber \sum_{k=1}^{n}ar^{k-1} textless ar^{n}
%\end{align}
\end{lemma}
\begin{proof}
%Suppose a geometric progression with first term $a$ and common ratio $r$ as follows:
%\begin{align}
 %   \nonumber a, ar, ar^2, ar^3, ar^4, ...
%\end{align}
%If $r \ge 2$ and $a\textgreater 0$, then the sum of first $n-1$ terms is less than the value of the $n$-th term.
%\begin{align}
 %   \nonumber \sum_{k=1}^{n-1}ar^{k-1} %\textless ar^{n-1}
%\end{align}
For every $n$, the sum of the first $n$ terms is $\sum_{k=1}^{n}ar^{k-1}$ and the ($n+1$)-th term is $ar^{n}$.
\begin{align}
    \nonumber \sum_{k=1}^{n}ar^{k-1}&=\frac{a(r^{n}-1)}{r-1}\\
    \nonumber &=\frac{ar^{n}-a}{r-1}
\end{align}
\begin{align}
    \nonumber &r\ge 2 \implies r-1 \ge 1 \implies \frac{1}{r-1} \le 1
\end{align}
Thus, 
\begin{align}
    \nonumber \sum_{k=1}^{n}ar^{k-1}\le ar^{n}-a \textless ar^{n}
\end{align} 
\end{proof}
\begin{lemma}[\textbf{union bound}]\label{union}
Suppose $\{A_i\}_{i=1}^n$ is a collection of events. Then, $Pr[\cup_{i=1}^{n}A_i]\le\sum_{i=1}^{n}Pr[A_i]$.
\end{lemma}
%\begin{lemma}\label{Markov}
%Suppose $X$ is a random variable taking non-negative values. For all $\alpha \textgreater 0$, $Pr[X\textgreater \alpha ]\le \frac{E[X]}{\alpha}$ .
%\end{lemma}
%\begin{proof}
%$E[X]=Pr[X\textgreater \alpha] \cdot E[X|X\textgreater \alpha ]+Pr[X\le \alpha]\cdot E[X|X \le \alpha]$\\
%Observe that $E[X|X\textgreater \alpha ]\textgreater \alpha$, $Pr[X\le \alpha]\ge 0$.
%Since $X$ is a random variable taking non-negative values, $E[X|X \le \alpha] \ge 0$.
%Hence $E[X]\ge Pr[X\textgreater \alpha] \cdot \alpha $. Since $\alpha \textgreater 0$, we have $Pr[X\textgreater \alpha ]\le \frac{E[X]}{\alpha}$.
%\end{proof}
%\begin{lemma}\label{Chebyshev}
%Suppose $X$ is a random variable with expectation $E[X]$. Then for all $\alpha \textgreater 0$, $Pr[X\textless E[X]-\alpha] \le \frac{var[X]}{\alpha^2}$, where $var[X]$ is the variance of $X$.
%\end{lemma}
%\begin{proof}
%Since $Pr[X\textless E[X]-\alpha]\le Pr[|X-E[X]|\textgreater \alpha]$, it suffices to prove $Pr[|X-E[X]|\textgreater \alpha] \le \frac{var[X]}{\alpha^2}$.
%Observe that $Pr[|X-E[X]|\textgreater \alpha]=Pr[(X-E[X])^2\textgreater \alpha^2]$. Since $(X-E[X])^2$ takes non-negative values and $\alpha \textgreater 0$ implies $\alpha^2 \textgreater 0$, by Lemma \ref{Markov}, we have $Pr[(X-E[X])^2\textgreater \alpha^2]\le \frac{E[(X-E[X])^2]}{\alpha^2}=\frac{var[X]}{\alpha^2}$.
%\end{proof}
\subsection {Randomized Algorithm and Analysis}
In this section, first, we will give a randomized algorithm for graph with no short cycles problem, in which construct an $n$-vertex random graph $G_{n,p}$ with no cycles of length $l$ or less. Then analyze the algorithm and show that if $n\ge 2^{l+2}$, there exists graph $G_{n,p}$ that has at least $\Omega (n^{1+\frac{1}{l-1}})$ edges. The above can be summarized by the following theorem.
\begin{theorem}
For each $l\ge 3$, and $n\ge 2^{l+2}$, there exists an $n$-vertex graph with no cycles of length $l$ or less that has at least $\Omega (n^{1+\frac{1}{l-1}})$ edges.
\end{theorem}
\subsubsection{Randomized Algorithm}
Construct an $n$-vertex random graph $G_{n,p}$ with no cycles of length $l$ or less as follows. 
\begin{enumerate}
    \item Let $V$ be the set of $n$ vertices. For each unordered pair $\{u,v\}\in \tbinom{V}{2}$, independently add an edge between $u$ and $v$ with probability $p$ ($p\ge \frac{2}{n}$).
    \item Pick one edge in every cycle that of length $l$ or less and remove it.
\end{enumerate}

\subsubsection{Analysis} 
Let $X$ be the number of edges in $G_{n,p}$. Let $Y_i$ be the number of length-$i$ cycles in $G_{n,p}$. Let $Y:=\sum_{i=3}^{l}Y_i$.
For each cycle of length $l$ or less, pick one edge and remove it.
Then we can find an $n$-vertex graph with no cycles of length $l$ or less that has at least $Z:=X-Y$ edges.By the method mentioned above of constructing graph $G_{n,p}$, we have $E[X]=\tbinom{n}{2}p$, $var[X]=\tbinom{n}{2}p(1-p)$, $E[Y_i]=\binom{n}{i}\frac{(i-1)!}{2}p^i$. By Lemma \ref{linearity}, we have $E[Y]=\sum_{i=3}^{l}E[Y_i]$.
\begin{claim}
$E[Y]\textless (np)^l$
\end{claim}
\begin{proof}
\begin{align}
\nonumber E[Y]&=\sum_{i=3}^{l} \binom{n}{i}\frac{(i-1)!}{2}p^i\\
\nonumber &=\sum_{i=3}^{l} \frac{n!}{i!(n-i)!}\cdot\frac{(i-1)!}{2}p^i\\
\nonumber &=\sum_{i=3}^{l} \frac{n!}{(n-i)!}\cdot\frac{1}{2i}p^i\\
\nonumber &\textless \sum_{i=3}^{l}n^i\cdot\frac{1}{2i}p^i\\
\nonumber &\le \frac{1}{6} \sum_{i=3}^{l}(np)^i
\end{align}
Since $p \ge \frac{2}{n}$, we have $np\ge 2$.
By Lemma \ref{gp}, we have:
\begin{align}
    \nonumber E[Y]& \textless   \frac{1}{6}\sum_{i=3}^{l}(np)^i \\
    \nonumber & \textless \frac{1}{6}\sum_{i=1}^{l}(np)^i \\
    \nonumber & \textless \frac{1}{3}(np)^l\\
    \nonumber & \textless (np)^l
\end{align}
\end{proof}
Suppose we can show the following:
\begin{enumerate}
    \item For some $\alpha \textgreater 0$, $Pr[X\textless E[X]-\alpha] \textless \frac{1}{2}$
    \item For some $\beta \textgreater 0$, $Pr[Y \textgreater \beta]\textless \frac{1}{2}$
\end{enumerate}
By Lemma \ref{union}, we have that with non-zero probability neither events happen, that is $Pr[(X\ge E[X]-\alpha)\cap (Y \le \beta)]\textgreater 0$. Then it shows that there exists an $n$-vertex graph with no cycles of length $l$ or less that has at least $E[X]-\alpha-\beta$ edges. 
Then we will show how to bound $E[X]-\alpha-\beta$.
Choose an appropriate value of $p$ such that $E[X] \ge 2E[Y]$. Since $E[X]=\tbinom{n}{2}p$ and $E[Y]\textless (np)^l$, we have to satisfy $\tbinom{n}{2}p \ge 2(np)^l$. Thus, $p\le  \sqrt[l-1]{\frac{1-\frac{1}{n}}{4n^{l-2}}}$. Since $l\ge 3$, $n\ge 2^{l+2}$ and $p\ge \frac{2}{n}$, we can set $p:=\sqrt[l-1]{\frac{1}{8n^{l-2}}}$. Check that $E[X]=\Theta (n^\frac{l}{l-1})$.% Since $E[X] \ge 2E[Y]$ and $E[Y]\le (np)^l$, we have $E[Y]=O(n^\frac{l}{l-1})$.

For $\alpha \textgreater 0$, by 
%Lemma \ref{Chebyshev}
Chebyshev's inequality
, we have:
\begin{align}
    \nonumber Pr[X\textless E[X]-\alpha]&\le Pr[|X-E[X]|\textgreater \alpha]\\ 
    \nonumber &\le \frac{var[X]}{\alpha^2}\\
    \nonumber &=\frac{\tbinom{n}{2}p(1-p)}{\alpha^2}
\end{align}
What we need is to satisfy $Pr[X\textless E[X]-\alpha] \textless \frac{1}{2}$, so $\frac{\tbinom{n}{2}p(1-p)}{\alpha^2}\le \frac{1}{2} $. Thus $\alpha \ge \sqrt{2\tbinom{n}{2}p(1-p)}$. We can set $\alpha = \sqrt{2\tbinom{n}{2}p}=\Theta(n^{\frac{l}{2l-2}})$.

For $\beta \textgreater 0$, by 
%Lemma \ref{Markov}
Markov's inequality
, we have: 
\begin{align}
    \nonumber Pr[Y \textgreater \beta] &\le \frac{E[Y]}{\beta}\\
    \nonumber & \le \frac{(np)^l}{\beta} 
\end{align}
Since we have to satisfy $Pr[Y \textgreater \beta]\textless \frac{1}{2}$, we have $\frac{(np)^l}{\beta} \le \frac{1}{2}$. Thus $\beta \ge 2(np)^l$. We can set $\beta = 2(np)^l = \Theta (n^{\frac{l}{l-1}})$.
Then,
\begin{align}
    \nonumber E[X]-\alpha-\beta = \Theta (n^\frac{l}{l-1}) - \Theta(n^{\frac{l}{2l-2}}) - \Theta (n^{\frac{l}{l-1}})=\Omega (n^{1+\frac{1}{l-1}})
\end{align}
In addition, we can bound the expectation of $Z$.
Since $E[Z]=E[X-Y]=E[X]-E[Y]$ and $E[X]\ge 2E[Y]$, we have:
\begin{align}
    \nonumber E[Z]\ge E[X]-\frac{1}{2}E[X]=\frac{1}{2}E[X]=\Omega (n^\frac{l}{l-1})
\end{align}
\subsection{Deterministic Algorithm and Analysis}
In this section, we will give a deterministic algorithm, which can construct an $n$-graph $G$ without cycles of length $l$ or less. 
Recall that $Z$ is the number of edges in the final graph.
Then we will analyze the edges of graph $G$ is at least $E[Z]$. 
Finally, we will analyze the running time of the deterministic algorithm.
\subsubsection{Deterministic Algorithm}
Denote $[i]:=\{0,1,2,...,i-1\}$ and $[0]=\emptyset$.
In graphs with no short cycles problem, for each unordered pair $\{u, v\} \in \tbinom{V}{2}$, we need to decide whether add an edge between vertex $u$ and vertex $v$. 
Totally, we have to make decision for $\tbinom{n}{2}$ random variables.
We denote the $\tbinom{n}{2}$ random variables as $R_0, R_1, ..., R_{\tbinom{n}{2}-1}$. %we use the following method to label all pairs. 
For $i\in [\tbinom{n}{2}]$, $R_i$ is considered as an edge to be selected in graph $G$.
Define $R_i$ to be $1$ if we add edge $R_i$ in graph $G$ and $0$ otherwise. 
Then, we can use $\sum_{i=0}^{\tbinom{n}{2}-1}R_i$ to denote the object value of the problem. The constraint is that there are no cycles of length $l$ or less.
For $i\in [\tbinom{n}{2}]$, let $R_{[i]}=r_{[i]}$ be a partial assignment where $r_{[i]}=1$ or $r_{[i]}=0$.
\begin{algorithm}
    \caption{Derandomization for Graphs with No Short Cycles}\label{alg}
    \KwIn{$n$ vertices: $V$} 
    \KwOut{a graph $G=(V,E)$} 
    {
        $E$=$\emptyset$;

        \For {$i=0$ to $\tbinom{n}{2}-1$}
        {
            \If{$E[Z|R_{[i]}=r_{[i]}\wedge R_i=1] \ge E[Z|R_{[i]}=r_{[i]}\wedge R_i=0]$}
            {
               $ E=E\cup R_i$
            }
        
        }
    }
\end{algorithm}
\subsubsection{Performance Guarantee}

Initially, when $i=0$, $[0]=\emptyset$, we have $E[Z|R_{[0]}=r_{[0]}]=E[Z]$.
Then we will show the relationship between values $E[Z|R_{[i]}=r_{[i]}]$ and  $E[Z|R_{[i+1]}=r_{[i+1]}]$. 
\begin{align}
    \nonumber E[Z|R_{[i]}=r_{[i]}]&=Pr[R_i=1|R_{[i]}=r_{[i]}]\cdot E[Z|R_{[i]}=r_{[i]}\wedge R_i=1]\\
    \nonumber &~~~~+Pr[R_i=0|R_{[i]}=r_{[i]}]\cdot E[Z|R_{[i]}=r_{[i]}\wedge R_i=0]
\end{align}
Thus we have either $E[Z|R_{[i]}=r_{[i]}\wedge R_i=1]$ or $E[Z|R_{[i]}=r_{[i]}\wedge R_i=0]$ is not less than  $E[Z|R_{[i]}] $. In algorithm \ref{alg}, we choose the larger one. So we have $E[Z|R_{[i+1]}=r_{[i+1]}] \ge E[Z|R_{[i]}=r_{[i]}]$ where $i\in [\tbinom{n}{2}-1]$.
Thus, the algorithm \ref{alg} can guarantee that the output graph $G$ has edges at least $E[Z]$.
\subsubsection{Time Complexity }
In Algorithm \ref{alg}, we decide whether to add edge $R_i$ in $G$ one by one. There are totally $\tbinom{n}{2}$ edges to be selected.
When we consider whether to add edge $R_i$, we have to compute two expectations, $E[Z|R_{[i]}=r_{[i]}\wedge R_i=1]$ and $E[Z|R_{[i]}=r_{[i]}\wedge R_i=0]$. Then I will show how to compute $E[Z|R_{[i]}=r_{[i]}]$.
\begin{align}
E[Y_i]=\binom{n}{i}\frac{(i-1)!}{2}p^i    
\end{align}

\bibliographystyle{alpha}
\bibliography{reference}
\end{document}
