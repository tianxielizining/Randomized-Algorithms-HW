\section{Hoeffding's Inequality}
\noindent 0. \blue{Assume that there is a randomized algorithm $A$ that is suitable for a yes/no problem.
And algorithm $A$ can return the correct answer with probability $p\in(\frac{1}{2},1)$. 
How to design a randomized algorithm based on $A$ with failure probability at most $\delta$? } \\
\textbf{Algorithm:}\\
\begin{enumerate}
    \item  Repeat algorithm $A$ for $2n+1$ times and we can get $2n+1$ outputs.
    Denote the $2n+1$ outputs as $X_0,X_1,...,X_{2n}$. Here, $n=\lceil \frac{\frac{-8p\cdot \ln \delta }{(2p-1)^2} -1}{2}\rceil $.
    \item Let $X$ be the output of the algorithm. 
    If $\frac{\sum_{i=1}^{2n+1}X_i}{2n+1}>\frac{1}{2}$, return $X=1$. Otherwise, return $X=0$.
\end{enumerate}
\textbf{Analysis:}\\
Let $C$ be the correct answer, and $C\in\{0,1\}$.
For each $i\in [2n+1]$, $X_i\in \{0, 1\}$. 
For each $i\in [2n+1]$, use $Y_i$ to denote the correctness of $X_i$.
If $X_i=C$, $Y_i=1$; if $X_i\neq C$, $Y_i=0$. 
Since the algorithm $A$ can return the correct answer with probability $p$, 
we have for each $i \in [2n+1]$, $E[Y_i]=p$.
Let $Z=\sum_{i\in[2n+1]}Y_i$, and then, we have $E[Z]=(2n+1)p$.\\
The algorithm will fail if and only if $Z$ is lesser than $\frac{1}{2}(2n+1)$. 
And we have to make sure that the failure probability at most $\delta$, so we have:
\begin{align}
    \nonumber  Pr[Z<\frac{1}{2}(2n+1)]\le \delta
\end{align}
By Chernoff Bound, we have:
\begin{align}
    \nonumber Pr[Z \le (1-\epsilon) E[Z]]\le e^{-\frac{1}{2}\epsilon^2(2n+1)p}
\end{align}
Let $(1-\epsilon) E[Z]= \frac{1}{2}(2n+1)$. Since $E[Z]=(2n+1)p$, we can compute $\epsilon=1-\frac{1}{2p}$.
Since $p\in (\frac{1}{2},1)$, the $\epsilon \in (0,1) $.
Thus, we have $Pr[Z \le \frac{1}{2}(2n+1)]\le e^{\frac{-(2p-1)^2(2n+1)}{8p}}$.
Since $Pr[Z < \frac{1}{2}(2n+1)]\le Pr[Z \le \frac{1}{2}(2n+1)]$ and we have to satisfy $Pr[Z<\frac{1}{2}(2n+1)]\le \delta$, then:
\begin{align}
    \nonumber e^{\frac{-(2p-1)^2(2n+1)}{8p}} \le \delta \implies 2n+1 \ge \frac{-8p\cdot \ln \delta }{(2p-1)^2}
\end{align}
\\
\noindent 1. \blue{\textbf{Integration by Sampling.}
Suppose we are given an integrable function 
$f : [0, 1]\rightarrow [0, M ]$,
and we wish to estimate the integral $I (f ) := \int_{0}^{1} f (x) \,dx$.
We only have \textbf{black box access} to the function $f$: 
this means that we are given a box such that 
when we provide it with a real number $x$, 
the box returns the value $f(x)$.
Moreover, we assume the real computation model. 
In particular, we assume that storing a real number takes constant space, 
and basic arithmetic and comparison operator ($\le$) take constant time.
Suppose we are also given a random number generator \textbf{Rand[0,1] }
that returns a number uniformly at random from $[0,1]$, 
and subsequent runs of \textbf{Rand[0,1]} are independent.
The goal is to design an algorithm that given black box access to a function 
$f : [0, 1]\rightarrow [0, M ]$ and parameters $0 < \epsilon ,\delta < 1$, 
return an estimate of $I(f)$ with additive error at most $\epsilon$ 
and failure probability at most $\delta$.
\\
(a) Show that this is not achievable by a deterministic algorithm. 
In particular, show that for any deterministic algorithm $A$, 
there is some function $f$ such that the algorithm $A$
returns an answer with additive error $\frac{M}{2}$ .
}

\noindent \blue{(b) Using the random generator \textbf{Rand[0,1]}, 
design a randomized algorithm to achieve the desired goal.
Give the number of black box accesses to the function $f$ 
and the number of accesses to \textbf{Rand[0,1]} used by your algorithm.}\\
The randomized algorithm is as follows.
\begin{enumerate}
    \item Use \textbf{Rand[0.1]} for $n$ times and get $n$ random numbers.
    Here, $n=\lceil \frac{M^2(\ln 2-\ln \delta)}{2\epsilon^2}\rceil $.
    Denote the $n$ random numbers as $X_0, X_1, ..., X_{n-1}$.
    \item For each $X_i$, $i\in [n]$, use \textbf{black box access} to get $f(X_i)$.
    Let $Y=\sum_{i\in n}f(X_i)$.
    \item Return $\frac{Y}{n}$ as the estimate of $I(f)$.
\end{enumerate}
\textbf{Analysis:}\\
Since \textbf{Rand[0,1]} can generate random number independent,
$X_1, X_2,..., X_{n-1}$ are independent.
And then, $f(X_1), f(X_2),..., f(X_{n-1})$ are independent.
Furthermore, all of the $f(X_i)$s take values from the interval $[0, M]$.
$Y=\sum_{i\in n}f(X_i)$, by definition, we have $E[\frac{Y}{n}]=I(f)$.
By Hoeffding's Inequality, we have:
\begin{align}
    \nonumber Pr[|\frac{Y}{n}-I(f)|\ge\epsilon]=Pr[|\frac{Y}{n}-E[\frac{Y}{n}]|\ge\epsilon]
    =Pr[|Y-E[Y]|\ge n\epsilon]\le 2e^{\frac{-2n\epsilon^2}{M^2}}
\end{align}
What we need to satisfy is $Pr[|\frac{Y}{n}-I(f)|\ge\epsilon]\le \delta$, then,
\begin{align}
    \nonumber 2e^{\frac{-2n\epsilon^2}{M^2}} \le \delta \implies n\ge \frac{M^2(\ln 2-\ln \delta)}{2\epsilon^2}
\end{align}

