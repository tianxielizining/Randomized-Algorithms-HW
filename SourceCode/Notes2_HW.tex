\section{Graphs with No Short Cycles}
\begin{definition}
An undirected graph $G = (V,E)$ contains a cycle of length $l$ if there are $l$ vertices $v_1,v_2,...,v_l$ such that all $l$ edges $\{v_1,v_2\},\{v_2,v_3\},...\{v_{l−1},v_l\},\{v_l,v_1\} \in E$ are present. The minimum length of a cycle is $3$.
\end{definition}
\noindent\textbf{Question.} Suppose a graph has no cycles of length $l$ or less. What is the maximum number of edges that it can have?

We will give a randomized algorithm and a deterministic algorithm in the following subsections. Both algorithms guarantee the number of edges.
\subsection{Preliminaries}
%The following lemmas and definition are useful for analysis.

\begin{lemma}[\textbf{linearity of expectation}]\label{linearity}
Suppose for each $1 \le i \le n$, $X_i$ is a random variable and $a_i$ is a real number. Then, $E[\sum_{i=1}^na_iX_i]=\sum_{i=1}^{n}a_iE[X_i]$.
\end{lemma}
\begin{lemma}\label{gp}
Given a geometric progression with first term $a > 0$ and common ratio $r \ge 2$, for every $n$, the sum of the first $n$ terms is less than the ($n+1$)-th term.
%\begin{align}
 %  \nonumber \sum_{k=1}^{n}ar^{k-1} textless ar^{n}
%\end{align}
\end{lemma}
\begin{proof}
%Suppose a geometric progression with first term $a$ and common ratio $r$ as follows:
%\begin{align}
 %   \nonumber a, ar, ar^2, ar^3, ar^4, ...
%\end{align}
%If $r \ge 2$ and $a\textgreater 0$, then the sum of first $n-1$ terms is less than the value of the $n$-th term.
%\begin{align}
 %   \nonumber \sum_{k=1}^{n-1}ar^{k-1} %\textless ar^{n-1}
%\end{align}
For every $n$, the sum of the first $n$ terms is $\sum_{k=1}^{n}ar^{k-1}$ and the ($n+1$)-th term is $ar^{n}$.
\begin{align}
    \nonumber \sum_{k=1}^{n}ar^{k-1}&=\frac{a(r^{n}-1)}{r-1}\\
    \nonumber &=\frac{ar^{n}-a}{r-1}
\end{align}
\begin{align}
    \nonumber &r\ge 2 \implies r-1 \ge 1 \implies \frac{1}{r-1} \le 1
\end{align}
Thus, 
\begin{align}
    \nonumber \sum_{k=1}^{n}ar^{k-1}\le ar^{n}-a < ar^{n}
\end{align} 
\end{proof}
\begin{lemma}[\textbf{union bound}]\label{union}
Suppose $\{A_i\}_{i=1}^n$ is a collection of events. Then, $\Pr[\cup_{i=1}^{n}A_i]\le\sum_{i=1}^{n}\Pr[A_i]$.
\end{lemma}
\subsection {Randomized Algorithm and Analysis}\label{RA}
In this section, first, we will give a randomized algorithm for graph with no short cycles problem, in which construct an $n$-vertex random graph $G_{n,p}$ with no cycles of length $l$ or less. Then analyze the algorithm and show that if $n\ge 2^{l+2}$, there exists graph $G_{n,p}$ that has at least $\Omega (n^{1+\frac{1}{l-1}})$ edges. The above can be summarized by the following theorem.
\begin{theorem}
For each $l\ge 3$, and $n\ge 2^{l+2}$, there exists an $n$-vertex graph with no cycles of length $l$ or less that has at least $\Omega (n^{1+\frac{1}{l-1}})$ edges.
\end{theorem}
\subsubsection{Randomized Algorithm}
Construct an $n$-vertex random graph $G_{n,p}$ with no cycles of length $l$ or less as follows. 
\begin{enumerate}
    \item Let $V$ be the set of $n$ vertices. For each unordered pair $\{u,v\}\in \tbinom{V}{2}$, independently add an edge between $u$ and $v$ with probability $p$ ($p\ge \frac{2}{n}$).
    \item Pick one edge in every cycle that of length $l$ or less and remove it.
\end{enumerate}

\subsubsection{Analysis}\label{Analysis} 
Let $X$ be the number of edges in $G_{n,p}$. Let $Y_i$ be the number of length-$i$ cycles in $G_{n,p}$. Let $Y:=\sum_{i=3}^{l}Y_i$.
For each cycle of length $l$ or less, pick one edge and remove it.
Then we can find an $n$-vertex graph with no cycles of length $l$ or less that has at least $Z:=X-Y$ edges.By the method mentioned above of constructing graph $G_{n,p}$, we have $E[X]=\tbinom{n}{2}p$, $var[X]=\tbinom{n}{2}p(1-p)$, $E[Y_i]=\binom{n}{i}\frac{(i-1)!}{2}p^i$. By Lemma \ref{linearity}, we have $E[Y]=\sum_{i=3}^{l}E[Y_i]$.
\begin{claim}
$E[Y]<(np)^l$
\end{claim}
\begin{proof}
\begin{align}
\nonumber E[Y]&=\sum_{i=3}^{l} \binom{n}{i}\frac{(i-1)!}{2}p^i\\
\nonumber &=\sum_{i=3}^{l} \frac{n!}{i!(n-i)!}\cdot\frac{(i-1)!}{2}p^i\\
\nonumber &=\sum_{i=3}^{l} \frac{n!}{(n-i)!}\cdot\frac{1}{2i}p^i\\
\nonumber &<\sum_{i=3}^{l}n^i\cdot\frac{1}{2i}p^i\\
\nonumber &\le \frac{1}{6} \sum_{i=3}^{l}(np)^i
\end{align}
Since $p \ge \frac{2}{n}$, we have $np\ge 2$.
By Lemma \ref{gp}, we have:
\begin{align}
    \nonumber E[Y]& <  \frac{1}{6}\sum_{i=3}^{l}(np)^i \\
    \nonumber & < \frac{1}{6}\sum_{i=1}^{l}(np)^i \\
    \nonumber & < \frac{1}{3}(np)^l\\
    \nonumber & < (np)^l
\end{align}
\end{proof}
Suppose we can show the following:
\begin{enumerate}
    \item For some $\alpha > 0$, $\Pr[X< E[X]-\alpha] <\frac{1}{2}$
    \item For some $\beta > 0$, $\Pr[Y > \beta]< \frac{1}{2}$
\end{enumerate}
By Lemma \ref{union}, we have that with non-zero probability neither events happen, that is $\Pr[(X\ge E[X]-\alpha)\cap (Y \le \beta)]> 0$. Then it shows that there exists an $n$-vertex graph with no cycles of length $l$ or less that has at least $E[X]-\alpha-\beta$ edges. 
Then we will show how to bound $E[X]-\alpha-\beta$.
Choose an appropriate value of $p$ such that $E[X] \ge 2E[Y]$. Since $E[X]=\tbinom{n}{2}p$ and $E[Y]<(np)^l$, we have to satisfy $\tbinom{n}{2}p \ge 2(np)^l$. Thus, $p\le  \sqrt[l-1]{\frac{1-\frac{1}{n}}{4n^{l-2}}}$. Since $l\ge 3$, $n\ge 2^{l+2}$ and $p\ge \frac{2}{n}$, we can set $p:=\sqrt[l-1]{\frac{1}{8n^{l-2}}}$. Check that $E[X]=\Theta (n^\frac{l}{l-1})$.% Since $E[X] \ge 2E[Y]$ and $E[Y]\le (np)^l$, we have $E[Y]=O(n^\frac{l}{l-1})$.

For $\alpha > 0$, by 
%Lemma \ref{Chebyshev}
Chebyshev's inequality
, we have:
\begin{align}
    \nonumber \Pr[X< E[X]-\alpha]&\le \Pr[|X-E[X]|> \alpha]\\ 
    \nonumber &\le \frac{var[X]}{\alpha^2}\\
    \nonumber &=\frac{\tbinom{n}{2}p(1-p)}{\alpha^2}
\end{align}
What we need is to satisfy $\Pr[X< E[X]-\alpha] < \frac{1}{2}$, so $\frac{\tbinom{n}{2}p(1-p)}{\alpha^2}\le \frac{1}{2} $. Thus $\alpha \ge \sqrt{2\tbinom{n}{2}p(1-p)}$. We can set $\alpha = \sqrt{2\tbinom{n}{2}p}=\Theta(n^{\frac{l}{2l-2}})$.

For $\beta > 0$, by 
%Lemma \ref{Markov}
Markov's inequality
, we have: 
\begin{align}
    \nonumber \Pr[Y >\beta] &\le \frac{E[Y]}{\beta}\\
    \nonumber & \le \frac{(np)^l}{\beta} 
\end{align}
Since we have to satisfy $\Pr[Y > \beta]< \frac{1}{2}$, we have $\frac{(np)^l}{\beta} \le \frac{1}{2}$. Thus $\beta \ge 2(np)^l$. We can set $\beta = 2(np)^l = \Theta (n^{\frac{l}{l-1}})$.
Then,
\begin{align}
    \nonumber E[X]-\alpha-\beta = \Theta (n^\frac{l}{l-1}) - \Theta(n^{\frac{l}{2l-2}}) - \Theta (n^{\frac{l}{l-1}})=\Omega (n^{1+\frac{1}{l-1}})
\end{align}
In addition, we can bound the expectation of $Z$.
Since $E[Z]=E[X-Y]=E[X]-E[Y]$ and $E[X]\ge 2E[Y]$, we have:
\begin{align}
    \nonumber E[Z]\ge E[X]-\frac{1}{2}E[X]=\frac{1}{2}E[X]=\Omega (n^\frac{l}{l-1})
\end{align}
\subsection{Deterministic Algorithm and Analysis}
In this section, we will give a deterministic algorithm, which can construct an $n$-vertex graph $G$ without cycles of length $l$ or less.
%Recall that $Z$ is the number of edges in the final graph.
The deterministic algorithm is a derandomization of the randomized algorithm in Section \ref{RA} by using conditional expectation.
After giving the deterministic algorithm, we will analyze
the performance guarantee and time complexity of this algorithm. 

Since the algorithm is related to the randomized algorithm in Section \ref{RA}, we will continue use notations that defined in \ref{RA} in this section.
In Section \ref{RA}, we defined random variables, $X$, $Y_i$, $Y$ and $Z$ as follows. 
$X$ represents the number of edges in $G_{n,p}$, which is constructed by independently add an edge between two vertices in $n$ vertices with probability $p$. 
$Y_i$ means the number of length-$i$ cycles in $G_{n,p}$. 
$Y$ indicates the number of cycles of length $l$ or less, thus $Y=\sum_{i=3}^{l}Y_i$.
$Z$ is the objective value of graphs with no short cycles problem. And it is a function of the random variables, $Z=X-Y=X-\sum_{i=3}^{l}Y_i$.
%we will analyze both the edges of graph $G$ is at least $E[Z]$. 
%Finally, we will analyze the running time of the deterministic algorithm.
\subsubsection{Deterministic Algorithm}
Denote $[i]:=\{0,1,2,...,i-1\}$ and $[0]=\emptyset$.
In graphs with no short cycles problem, there are $\tbinom{n}{2}$ candidate edges. 
We use $E:=[\tbinom{n}{2}]$ to denote all candidate edges. 
Using random variables $R_0, R_1, ..., R_{\tbinom{n}{2}-1}$ to denote whether edge $i$ is in graph $G$.
Define $R_i$ to be $1$ if we add edge $i$ in graph $G$ and $0$ otherwise. 
Then, we can use $\sum_{i=0}^{\tbinom{n}{2}-1}R_i$ to denote the object value of the problem. The constraint is that there are no cycles of length $l$ or less.
For $i\in [\tbinom{n}{2}]$, let $R_{[i]}:=r_{[i]}$ be a partial assignment, which means we have already assigned value to random variables $R_{[i]}$.


\begin{algorithm}
    \caption{Derandomization for Graphs with No Short Cycles}\label{alg}
    \KwIn{$n$ vertices: $V$, length $l$} 
    \KwOut{a graph $G=(V,E)$ with no cycles of length $l$ or less } 
    {
        $E$=$\emptyset$;

        \For {$i=0$ to $\tbinom{n}{2}-1$}
        {
            \If{$E[Z|R_{[i]}=r_{[i]}\wedge R_i=1] \ge E[Z|R_{[i]}=r_{[i]}\wedge R_i=0]$}
            {
               $r_i=1$
               
               $ E=E\cup \{i\}$
            }
            \Else
            {
                $r_i=0$
            }
        
        }
    }
\end{algorithm}
\subsubsection{Performance Guarantee}

Initially, when $i=0$, $[i]=\emptyset$, we have $E[Z|R_{[0]}=r_{[0]}]=E[Z]$.
Then we will show the relationship between values $E[Z|R_{[i]}=r_{[i]}]$ and  $E[Z|R_{[i+1]}=r_{[i+1]}]$. 
\begin{align}
    \nonumber E[Z|R_{[i]}=r_{[i]}]&=\Pr[R_i=1|R_{[i]}=r_{[i]}]\cdot E[Z|R_{[i]}=r_{[i]}\wedge R_i=1]\\
    \nonumber &~~~~+\Pr[R_i=0|R_{[i]}=r_{[i]}]\cdot E[Z|R_{[i]}=r_{[i]}\wedge R_i=0]
\end{align}
Thus we have either $E[Z|R_{[i]}=r_{[i]}\wedge R_i=1]$ or $E[Z|R_{[i]}=r_{[i]}\wedge R_i=0]$ is not less than  $E[Z|R_{[i]}=r_{[i]}] $. In algorithm \ref{alg}, we choose the larger one. So we have $E[Z|R_{[i+1]}=r_{[i+1]}] \ge E[Z|R_{[i]}=r_{[i]}]$ where $i\in [\tbinom{n}{2}-1]$.
Thus, the algorithm \ref{alg} can guarantee that the output graph $G$ has edges at least $E[Z]$.
\subsubsection{Time Complexity }
In Algorithm \ref{alg}, we decide whether to add edge $R_i$ in $G$ one by one. There are totally $\tbinom{n}{2}$ edges to be selected.
When we consider whether to add edge $R_i$, we have to compute two expectations, $E[Z|R_{[i]}=r_{[i]}\wedge R_i=1]$ and $E[Z|R_{[i]}=r_{[i]}\wedge R_i=0]$. Then I will show how to compute $E[Z|R_{[i]}=r_{[i]}]$.
\begin{align}
    \nonumber E[Z|R_{[i]}=r_{[i]}]&=E[X-Y|R_{[i]}=r_{[i]}]\\
    \nonumber &=E[X|R_{[i]}=r_{[i]}]-E[Y|R_{[i]}=r_{[i]}]\\
    \nonumber &=\sum_{x=0}^{i-1}r_x + (\binom{n}{2}-i)\cdot p -E[Y|R_{[i]}=r_{[i]}]
\end{align}
Since there are $\binom{n}{2}$ layers of loops, the condition expectation needs to be calculated twice in each layer of the loop, we can get the running time of algorithm \ref{alg} is $O()$.
