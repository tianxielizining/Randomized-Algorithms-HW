\section{A Proof using Stirling's Formula}
\begin{lemma}[\textbf{Stirling's formula}\cite{stirling}]\label{Stirling}
 For $n=1,2,3,...$
\begin{align}
    \nonumber \sqrt{2\pi}n^{n+\frac{1}{2}}e^{-n}e^{\frac{1}{12n+1}}< n! < \sqrt{2\pi}n^{n+\frac{1}{2}}e^{-n}e^{\frac{1}{12n}} 
\end{align}
\end{lemma}
\begin{definition}[\textbf{Euler's number}\cite{e}]\label{e}
The number $e$ is a mathematical constant approximately equal to $2.71828$. It is the limit of $(1+\frac{1}{n})^n$ as $n$ approaches infinity. The number $e$ is the unique real number such that
\begin{align}
    \nonumber (1+\frac{1}{x})^{x} < e <(1+\frac{1}{x})^{x+1}
\end{align}
\end{definition}
\begin{lemma}\label{gp}
Given a geometric progression with first term $a> 0$ and common ratio $r \ge 2$, for every $n$, the sum of the first $n$ terms is less than the ($n+1$)-th term.
%\begin{align}
 %  \nonumber \sum_{k=1}^{n}ar^{k-1} textless ar^{n}
%\end{align}
\end{lemma}
\begin{proof}
%Suppose a geometric progression with first term $a$ and common ratio $r$ as follows:
%\begin{align}
 %   \nonumber a, ar, ar^2, ar^3, ar^4, ...
%\end{align}
%If $r \ge 2$ and $a\textgreater 0$, then the sum of first $n-1$ terms is less than the value of the $n$-th term.
%\begin{align}
 %   \nonumber \sum_{k=1}^{n-1}ar^{k-1} %\textless ar^{n-1}
%\end{align}
For every $n$, the sum of the first $n$ terms is $\sum_{k=1}^{n}ar^{k-1}$ and the ($n+1$)-th term is $ar^{n}$.
\begin{align}
    \nonumber \sum_{k=1}^{n}ar^{k-1}&=\frac{a(r^{n}-1)}{r-1}\\
    \nonumber &=\frac{ar^{n}-a}{r-1}
\end{align}
\begin{align}
    \nonumber &r\ge 2 \implies r-1 \ge 1 \implies \frac{1}{r-1} \le 1
\end{align}
Thus, 
\begin{align}
    \nonumber \sum_{k=1}^{n}ar^{k-1}\le ar^{n}-a < ar^{n}
\end{align} 
\end{proof}


\begin{claim}
$\sum_{i=3}^{n}\tbinom{n}{i}p^i<(np)^l$
\end{claim}
\begin{proof}
Consider $\tbinom{n}{i}$, where $i\in\{3,4,5,...,l\}$. By Lemma \ref{Stirling}, we have:
\begin{align}
   \nonumber\tbinom{n}{i}&=\frac{n!}{i!(n-i)!}\\
    \nonumber&< \frac{\sqrt{2\pi}n^{n+\frac{1}{2}}e^{-n}e^{\frac{1}{12n}}}{\sqrt{2\pi}i^{i+\frac{1}{2}}e^{-i}e^{\frac{1}{12i+1}}\sqrt{2\pi}(n-i)^{(n-i)+\frac{1}{2}}e^{-(n-i)}e^{\frac{1}{12(n-i)+1}}}\\
    \nonumber &=\frac{1}{\sqrt{2\pi}}\cdot e^{\frac{1}{12n}-\frac{1}{12i+1}}\cdot e^{-\frac{1}{12(n-i)-1}} \cdot n^{n+\frac{1}{2}} \cdot \frac{1}{i^{i+\frac{1}{2}}} \cdot \frac{1}{(n-i)^{n-i+\frac{1}{2}}}\\
    &=\frac{1}{\sqrt{2\pi}}\cdot e^{\frac{1}{12n}-\frac{1}{12i+1}}\cdot e^{-\frac{1}{12(n-i)-1}} \cdot \frac{n^i}{i^{i+\frac{1}{2}}} \cdot (\frac{n}{n-i})^{n-i+\frac{1}{2}}\label{a}
\end{align}
Since $n\ge 2^{l+2}$ and $l\ge i$, we have $12n > 12i+1$, which implies $\frac{1}{12n} < \frac{1}{12i+1}$. Then
\begin{align}
     e^{\frac{1}{12n}-\frac{1}{12i+1}} < 1 \label{b}
\end{align}
By Definition \ref{e}, bound $(\frac{n}{n-i})^{n-i+\frac{1}{2}}$ as follows.
\begin{align}
    \nonumber (\frac{n}{n-i})^{n-i+\frac{1}{2}}
    &= (\frac{n-i+i}{n-i})^{n-i+\frac{1}{2}}\\
    \nonumber&= (1+\frac{i}{n-i})^{n-i+\frac{1}{2}}\\
    \nonumber&= (1+\frac{1}{\frac{n-i}{i}})^{n-i+\frac{1}{2}}\\
    \nonumber&= ((1+\frac{1}{\frac{n-i}{i}})^{\frac{n-i}{i}})^{i+\frac{i}{2(n-i)}}\\
    \nonumber &< e^{i+\frac{i}{2(n-i)}}
\end{align}
Since $\tbinom{n}{i}=\tbinom{n}{n-i}$, 
we can assume $i\le \frac{n}{2}$. 
%the assume that $i\le \frac{n}{2}$ is reasonable. 
In order to bound $(\frac{n}{n-i})^{n-i+\frac{1}{2}}$, we have to bound $\frac{i}{2(n-i)}$
\begin{align}
   \nonumber i\le \frac{n}{2} 
   \implies n \ge 2i 
   %\implies n-i \ge i 
   \implies 2(n-i) \ge 2i \implies \frac{1}{2(n-i)} \le \frac{1}{2i} \implies  \frac{i}{2(n-i)} \le \frac{i}{2i} =\frac{1}{2}
\end{align}
Thus, 
\begin{align}
    (\frac{n}{n-i})^{n-i+\frac{1}{2}} <e^{i+\frac{1}{2}}\label{c}
\end{align}
By inequation \ref{a}, \ref{b} and \ref{c} we have:
\begin{align}
\nonumber \tbinom{n}{i} &<\frac{1}{\sqrt{2\pi}}\cdot 1\cdot e^{-\frac{1}{12(n-i)-1}} \cdot \frac{n^i}{i^{i+\frac{1}{2}}} \cdot e^{i+\frac{1}{2}}\\
\nonumber &= \frac{1}{\sqrt{2\pi}}\cdot e^{i+\frac{1}{2}-\frac{1}{12(n-i)-1}} \cdot \frac{n^i}{i^{i+\frac{1}{2}}}\\
\nonumber &< \frac{1}{\sqrt{2\pi}}\cdot e^{i+\frac{1}{2}} \cdot \frac{n^i}{i^{i+\frac{1}{2}}}\\
\nonumber &= \frac{1}{\sqrt{2\pi}}\cdot (\frac{e}{i})^{i+\frac{1}{2}}\cdot n^i
\end{align}
Since $i\ge 3$, we have $\frac{e}{i}< 1$. Thus,
\begin{align}
    \nonumber \tbinom{n}{i} < \frac{1}{\sqrt{2\pi}}\cdot n^i
\end{align}
Then, we have
\begin{align}
    \nonumber E[Y] &= \sum_{i = 3}^{l}{\binom{n}{i} p^i} 
    \\
    \nonumber &<\sum_{i=1}^{l}\frac{1}{\sqrt{2\pi}}n^ip^i\\
    \nonumber &= \frac{1}{\sqrt{2\pi}}\sum_{i=3}^{l}(np)^i
\end{align}
Since $p \ge \frac{2}{n}$, we have $np\ge 2$.
By Lemma \ref{gp}, we have:
\begin{align}
    \nonumber E[Y]& <   \frac{1}{\sqrt{2\pi}}\sum_{i=3}^{l}(np)^i \\
    \nonumber & < \frac{1}{\sqrt{2\pi}}\sum_{i=1}^{l}(np)^i \\
    \nonumber & < \frac{2}{\sqrt{2\pi}}(np)^l\\
    \nonumber & <(np)^l
\end{align}


\end{proof}