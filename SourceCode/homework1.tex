\section{Homework 1}
\blue{1. Prove the union bound: Suppose $\{A_i\}_{i=1}^n$ is a collection of events. Then, $Pr[\cup_{i=1}^{n}A_i]\le\sum_{i=1}^{n}Pr[A_i]$.}
\begin{proof}
Let $t \in [n]$.
Prove the above by induction on $t$.
Certainly the inequality holds for the base case where $t=1$, since $Pr[A_1]=Pr[A_1]$.
Suppose the inequality holds for $n=t-1$ and if I can prove that the inequality holds for $n=t$ then finish the proof.
\begin{align}
    \nonumber Pr[\cup_{i=1}^{n}A_i]&=Pr[\cup_{i=1}^{n-1}A_i\cup A_n]\\
    \nonumber&= Pr[\cup_{i=1}^{n-1}A_i]+Pr[A_n]-Pr[(\cup_{i=1}^{n-1}A_i)\cap A_n]\\
    \nonumber&\le \sum_{i=1}^{n-1}Pr[A_i]+Pr[A_n]\\
    \nonumber&=\sum_{i=1}^{n}Pr[A_i]
\end{align}
\end{proof}
\noindent\blue{2. Prove the linearity of expectation: Suppose for each $1 \le i \le n$, $X_i$ is a random variable and $a_i$ is a real number. Then, $E[\sum_{i=1}^na_iX_i]=\sum_{i=1}^{n}a_iE[X_i]$.}
\cite{linearity}
\begin{definition}[\cite{Rajeev1995Randomized}]\label{def1}
The expectation of a random variable $X$ with density function $p$ is defined as $E[X] = \sum_x xp(x)$, where the summation is over the range of $X$.
\end{definition}
\begin{lemma}\label{lemma2}
For any random variable $X$ and $Y$, $E[X+Y]=E[X]+E[Y]$.
\end{lemma}
\begin{proof}
Prove by definition \ref{def1}.
$\sum_x$ and $\sum_y$ means the summation is over the range of $X$ and $Y$.
\begin{align}
    \nonumber E[X+Y]&=\sum_x\sum_y(x+y)Pr[(X=x)\cap(Y=y)]\\
    \nonumber&=\sum_x\sum_yxPr[(X=x)\cap(Y=y)]+yPr[(X=x)\cap(Y=y)]\\
    \nonumber&=\sum_xx\sum_yPr[(X=x)\cap(Y=y)]+\sum_yy\sum_xPr[(X=x)\cap(Y=y)]\\
    \nonumber&=\sum_xxPr[X=x]+\sum_yyPr[Y=y]\\
    \nonumber&=\sum_xxp(x)+\sum_yyp(y)\\
    \nonumber&=E[X]+E[Y]
\end{align}
\end{proof}
\begin{lemma}\label{lemma3}
For any random variable $X$ and any constant $a \in \mathbb{R}$, $E[aX]=aE[X]$.
\end{lemma}
\begin{proof}
Prove by definition \ref{def1}.
$\sum_x$ means the summation is over the range of $X$.
\begin{align}
    \nonumber E[aX]&=\sum_xaxp(x) %~~~~\text{the summation is over the range of }X  
    \\
    \nonumber &=a\sum_xxp(x)% ~~~~\text{the summation is over the range of }X
    \\
    \nonumber &=aE[X]
\end{align}
\end{proof}
By combining Claim \ref{lemma2} and \ref{lemma3}, we have:
\begin{align}
    \nonumber E[\sum_{i=1}^na_iX_i]&=\sum_{i=1}^nE[a_iX_i]\\
    \nonumber&=\sum_{i=1}^{n}a_iE[X_i]
\end{align}
\blue{
3. Suppose $X$, $Y: \Omega \rightarrow \mathbb{Z}$ are independent random variables taking integer values. In this question, we shall prove the basic result $E[XY]=E[X]E[Y]$.}
\begin{proof}
Prove by definition \ref{def1}.
$\sum_x$ and $\sum_y$ means the summation is over the range of $X$ and $Y$.
\begin{align}
    \nonumber E[XY]&=\sum_x\sum_yxyPr[(X=x)\cap(Y=y)]\\
    \nonumber &=\sum_x\sum_yxyPr[X=x]Pr[Y=y]\\
    \nonumber&=\sum_xxPr[X=x]\sum_yyPr[Y=y]\\
    \nonumber&=\sum_xxp(x)\sum_yyp(y)\\
    \nonumber&=E[X]E[Y]
\end{align}
\end{proof}
\noindent\blue{4. Let $\phi$ be a $3$-CNF formula with $m$ clauses and $n$ variables. \\
(a) Let $Y$ be the random variable denoting the number of unsatisfied clauses. Compute $E[Y]$.\\
(b) Find a suitable upper bound for $Pr[Y\textgreater\frac{3m}{16}]$, and conclude that the randomized algorithm finds an assignment satisfying at least $\frac{13m}{16}$ clauses with probability at least $\frac{1}{3}$.}\\
\textbf{Problem description:} There are $n$ Boolean variables. $\phi(x_1,x_2,...,x_n)$ is denoted as $C_1 \vee C_2 \vee...\vee C_m$. Each clause $C_i$ is a disjunction of $3$ literals from $3$ different variables. A literal is either a variable or its negation. A clause is satisfied if at least one of its $3$
literals evaluates to TRUE. The goal is to find an assignment of the variables so that as many clauses as possible are satisfied.\\
\textbf{Randomized algorithm:} Independently for each variable, assign its value to be TRUE or FALSE, each with probability $\frac{1}{2}$.\\
(a)
Define $Y_i$ as a random variable which is related to clause $C_i$. $Y_i$ takes value 1 if the clause $C_i$ is unsatisfied and 0 otherwise. Then the number of unsatisfied clauses $Y=\sum_{i=1}^{m}Y_i$. For each clause $C_i$, there are $3$ independent Boolean variables. And the value of each Boolean variable is independently assigned TRUE or FALSE, each with probability $\frac{1}{2}$. Since for each clause $C_i$ is unsatisfied only if all the three literals evaluate to FALSE, for each $Y_i$, takes value $1$ with probability $\frac{1}{8}$ and takes value $0$ with probability $\frac{7}{8}$. By Lemma \ref{lemma2}:
\begin{align}
    \nonumber E[Y]&=E[\sum_{i=1}^{m}Y_i]\\
    \nonumber&=\sum_{i=1}^mE[Y_i]\\
    \nonumber&=m(1\cdot \frac{1}{8}+0 \cdot \frac{7}{8})\\
    \nonumber&=\frac{m}{8}
\end{align}
(b)By Markov's inequality, for all $\alpha \textgreater 0$ we have
\begin{align}
   \nonumber Pr[Y\ge \alpha]\le \frac{E[Y]}{\alpha}
\end{align}
If set $\alpha=\frac{3m}{16}$,we have
\begin{align}
    \nonumber Pr[Y\ge \frac{3m}{16}]\le \frac{m}{8}\cdot \frac{16}{3m}=\frac{2}{3}
\end{align}
Thus, 
\begin{align}
    \nonumber Pr[Y\textgreater \frac{3m}{16}] \le Pr[Y\ge \frac{3m}{16}]\le \frac{2}{3} ~\implies~Pr[Y\le \frac{3m}{16}]\ge \frac{1}{3}
\end{align}
Thus, the randomized algorithm finds an assignment satisfying at least $\frac{13m}{16}$ clauses with probability at least $\frac{1}{3}$.\\
\blue{5. Let $G=(V, E)$ be a graph. Recall the randomized algorithm mentioned in class for finding a cut $C\subset V$ for the graph $G$, namely, a point $v \in V$ is included in $C$ independently with probability $\frac{1}{2}$. Assume that to make this decision takes $1$ independent random bit for each point.\\
Let $E(C):=\{\{u,v\}\in E: u\in C , v\in V\setminus C\}$ be the edges in the cut. It is shown that $E[|E(C)|]=\frac{|E|}{2}$. The goal of this question is to design another randomized algorithm with better guarantees. Let $0\textless\epsilon\textless1$ and $0\textless \delta \textless 1$. We shall design a randomized algorithm that, with failure probability at most $\delta$, returns a cut $C$ such that $|E(C)| \ge (\frac{1-\epsilon}{2})\cdot |E|$.\\
(a) Given an upper bound on the failure probability that the above randomized procedure returns a cut such that the number of edges $|E(C)|$ is less than $(\frac{1-\epsilon}{2})\cdot |E|$.\\
(b) Show that by repeating the above randomized procedure, it is possible to obtain a better randomized algorithm with failure probability at most $\delta$. Compute the number of independent random bits used by your algorithm.}
\cite{maxcut}\cite{reverse}\\
(a)
\begin{align}
  \nonumber Pr[|E(C)|\le(\frac{1-\epsilon}{2})|E|]=Pr[|E|-|E(C)|\ge |E|-\frac{1-\epsilon}{2}|E|]
\end{align}
By Markov's inequality, we have:
\begin{align}
  \nonumber Pr[|E|-|E(C)|\ge |E|-\frac{1-\epsilon}{2}]\le \frac{E[|E|-|E(C)|]}{|E|-\frac{1-\epsilon}{2}|E|}
\end{align}
By Lemma \ref{lemma2}, we have:
\begin{align}
  \nonumber E[|E|-|E(C)|]=|E|-E[|E(C)|]=|E|-\frac{|E|}{2}=\frac{|E|}{2}
\end{align}
Thus,
\begin{align}
\nonumber Pr[|E(C)|\le(\frac{1-\epsilon}{2})|E|]\le \frac{\frac{|E|}{2}}{|E|-\frac{1-\epsilon}{2}|E|}=\frac{1}{1+\epsilon}
\end{align}
(b)The random algorithm is described as follows.\\
\begin{enumerate}[(i)]
    \item For each vertex $v\in V$, independently assign it a number 0 or 1, each with probability $\frac{1}{2}$. Then we can get a set of vertices $C$ which consists of the vertices with number 1.
    \item Independently repeat step (i) for $k$ times, here, \blue{$k=-\frac{2}{\epsilon}\ln \delta$}, which is the answer to the question. We can get $k$ cuts of graph $G$, named $C_1, C_2,...,C_k$.
    \item For each cut $C_i$, compute $|E(C_i)|$, which is the number of edges in cut $C_i$. Then pick $C_j$ from all $C_i$s with the maximum $|E(C_i)|$. And then output $C_j$ as the final result. 
\end{enumerate}
\textbf{Analysis:}
By (a), for any $C_i$ we have
\begin{align}
    \nonumber Pr[|E(C_i)|\le (\frac{1-\epsilon}{2})\cdot |E|]\le \frac{1}{1+\epsilon}
\end{align}
Since $C_j$ is with the maximum $|E(C_i)|$ among all $C_j$s, we have
\begin{align}
    \nonumber Pr[|E(C_j)|\le (\frac{1-\epsilon}{2})\cdot |E|]&=\prod_{i=1}^k Pr[|E(C_i)|\le (\frac{1-\epsilon}{2})\cdot |E|]\\
    \nonumber &\le (\frac{1}{1+\epsilon})^k
\end{align}
Since for $0\textless \epsilon \textless 1$, $1+\epsilon \ge e^{\frac{\epsilon}{2}}$, we have
\begin{align}
    \nonumber Pr[|E(C_j)|\le (\frac{1-\epsilon}{2})\cdot |E|]&\le e^{-\frac{\epsilon k}{2}}\\
    \nonumber&=e^{-\frac{\epsilon}{2}(-\frac{2}{\epsilon}\ln \delta)}\\
    \nonumber&=\delta
\end{align}
\blue{
6.(a) Suppose $X$ is a discrete random variable that takes only a countable number of non-negative values. Prove that $E[X]=\int_{0}^{\infty}Pr[X\ge t]\, dt$. (If you do not like to deal with infinite support, you may assume $X$ only takes a finite number of values.)\\
(b) Assume that for all non-negative random variables $X$, $E[X]=\int_{0}^{\infty}Pr[X\ge t]\, dt$. Derive a similar formula for $E[X]$ when $X$ is any real-valued random variable.}\\
(a) 
First assume discrete random variable $X$ only takes a finite number $n$ of non-negative values, $x_1, x_2,...,x_n$ with probability $p_{x1},p_{x2},...,p_{xn}$. Note that $\sum_{i=1}^{n}p_{xi}=1$. 
By definition, $E[X]=\sum_{i=1}^{n}x_ip_{xi}$.
For simplicity, assume $0\le x_1\textless x_2 \textless ... \textless x_n$.
Then we can get a piecewise function as follows.
\begin{equation}
\nonumber Pr[X\ge t]=\left\{
\begin{array}{rcl}
1 & & {0\le t \le x_1}\\
1-p_{x1} & & {x_1 \textless t \le x_2}\\
1-\sum_{i=1}^{2}p_{xi} & & {x_2 \textless t \le x_3}\\
1-\sum_{i=1}^{3}p_{xi} & & {x_3 \textless t \le x_4}\\
... & & ...\\
1-\sum_{i=1}^{n-1}p_{xi} & & {x_{n-1} \textless t \le x_n}\\
0 & & {x_n \le t}
\end{array} \right.
\end{equation}
Thus we have:
\begin{align}
  \nonumber\int_{0}^{\infty}Pr[X\ge t]\, dt &= (x_1-0)\cdot 1+(x_2-x_1) (1-p_{x1})+(x_3-x_2) (1-\sum_{i=1}^{2}p_{xi})+\\
  \nonumber&(x_4-x_3) (1-\sum_{i=1}^{3}p_{xi})+...+(x_n-x_{n-1})(1-\sum_{i=1}^{n-1}p_{xi})+0\cdot \infty\\
  \nonumber&=x_1(1-1+p_{x1})+x_2(1-p_{x1}-(1-\sum_{i=1}^2p_{x_i}))+x_3(1-\sum_{i=1}^2p_{x_i}-(1-\sum_{i=1}^3p_{x_i}))\\
  \nonumber&~~~~+...+x_n(1-\sum_{i=1}^{n-1}p_{x_i}-(1-\sum_{i=1}^np_{x_i}))\\
  \nonumber&=x_1p_{x1}+x_2p_{x2}+x_3p_{x3}+...+x_np_{xn}\\
  \nonumber&=E[X]
\end{align}
Then consider the situation when discrete random variable $X$ takes infinite number of non-negative values (but the number of values is still countable).
Assume $X$ can take values $0\le x_1\textless x_2 \textless x_3 \textless ...$ with probability $p_{x1},p_{x2},p_{x3},...$ and $\int_{0}^{\infty}Pr[X=t]\, dt=1$. By definition, $E[X]=\sum_{k=0}^{\infty}kPr[X=k]$ and $Pr[X\ge t]=\sum_{k=t}^{\infty}Pr[X=k]$. Thus, we have:
\begin{align}
  \nonumber\int_{0}^{\infty}Pr[X\ge t]\, dt &=\int_{0}^{\infty}\sum_{k=t}^{\infty}Pr[X=k] \, dt\\
  \nonumber&=\int_{0}^{x_1}\sum_{k=x_1}^{\infty}Pr[X=k] \, dt + \int_{x_1}^{x_2}\sum_{k=x_2}^{\infty}Pr[X=k] \, dt + \int_{x_2}^{x_3}\sum_{k=x_3}^{\infty}Pr[X=k] \, dt +...\\
  \nonumber&=(x_1-0)\sum_{k=x_1}^{\infty}Pr[X=k]+(x_2-x_1)\sum_{k=x_2}^{\infty}Pr[X=k]+(x_3-x_2)\sum_{k=x_3}^{\infty}Pr[X=k]+...\\
  \nonumber&=x_1(\sum_{k=x_1}^{\infty}Pr[X=k]-\sum_{k=x_2}^{\infty}Pr[X=k])+x_2(\sum_{k=x_2}^{\infty}Pr[X=k]-\sum_{k=x_3}^{\infty}Pr[X=k])+\\
  \nonumber&~~~~ x_3(\sum_{k=x_3}^{\infty}Pr[X=k]-\sum_{k=x_2}^{\infty}Pr[X=k])+...\\
  \nonumber&=x_1Pr[X=x_1]+x_2Pr[X=x_2]+x_3Pr[X=x_3]+...\\
 % \nonumber&=x_1p_{x1}+x_2p_{x2}+x_3p_{x3}+...\\
  \nonumber &=\sum_{k=0}^{\infty}kPr[X=k]\\
  \nonumber &=E[X]
\end{align}
\\
(b)
Let $X$ be a continuous real-valued random variable. The probability density function of $X$ is $f(x)=\lim_{h \to 0} \frac{1}{h}\cdot Pr[x\le X \le x+h]$. By definition, $\int_{\mathbb{R}}f(x)\, dx=1$, $E[X]=\int_{\mathbb{R}}xf(x)\, dx$. $E[X]$ can also be written as
$E[X]=-\int_{-\infty}^{0}Pr[X\le t]\, dt+\int_{0}^{\infty}Pr[X\ge t]\, dt$
\begin{proof}
\begin{align}
    \nonumber \int_{0}^{\infty}Pr[X\ge t]\, dt &=\int_{0}^{\infty}
    \int_{t}^{\infty}f(k)\, dk
    \, dt\\
    \nonumber&=\int_{0}^{\infty}
    kf(k)
    \, dk\\
    \nonumber&=\int_{0}^{\infty}
    xf(x)
    \, dx
\end{align}
\begin{align}
    \nonumber -\int_{-\infty}^{0}Pr[X\le t]\, dt &=-\int_{-\infty}^{0}
    \int_{-\infty}^{t}f(k)\, dk
    \, dt\\
    \nonumber&=-\int_{0}^{-\infty}
   \int_{t}^{-\infty}f(k)\, dk
    \, dt\\
    \nonumber&=-\int_{0}^{-\infty}
    kf(k)
    \, dk\\
    \nonumber&=\int_{-\infty}^{0}
    xf(x)
    \, dx
\end{align}
\begin{align}
    \nonumber E[X]&=\int_{\mathbb{R}}xf(x)\, dx\\
    \nonumber &=\int_{-\infty}^{0}xf(x)\, dx+\int_{0}^{\infty}xf(x)\, dx\\
    \nonumber &=-\int_{-\infty}^{0}Pr[X\le t]\, dt + \int_{0}^{\infty}Pr[X\ge t]\, dt
\end{align}
\end{proof}