\section{$\epsilon$-Nets, VC-dimension}
\noindent \blue{
\textbf{1. VC-dimension of Axis-aligned rectangles.}
\\(a) Prove that no $5$ points on the plane $\mathbb{R}^2$ can be shattered by the class $C$ of axis-aligned
rectangles (that map points inside a rectangle $1$ and otherwise $0$).
}
\begin{definition}
    Given a set $X$ and a class $C$ of boolean function on $X$, 
    a subset $S \subset X$ is said to be shattered by $C$, 
    if for all subsets $U$ of $S$, there exists $f\in C$ such that for all 
    $x\in U$, $f(x) = 1$ and for all $x\in S \setminus U$, $f(x)=0$.
\end{definition}
In this problem, $X:=\mathbb{R}^k$, $C:=C_k$ is the class where each function 
corresponds to an $k$-dimension axis-aligned rectangle that labels each point inside it $1$ and otherwise $0$.
For a set $S_k$ which is a subset of $X$ that is said to be shattered by $C_k$,
if for all subsets $U_k$ of $S_k$, there exists a $k$-dimension axis-aligned rectangle such that for each point in 
$U_k$, the point is inside the $k$-dimension axis-aligned rectangle 
and for each point in set $S_k$ but not in set $U_k$, the point is not inside the  $k$-dimension axis-aligned rectangle.

\begin{proof}
    Use $S_2$ to denote any set of $5$ distinct points on the plane $\mathbb{R}^2$. 
    Let $x_{\max}$ be the maximum $x$-coordinate among the $5$ points and 
    let $x_{\min}$ be the minimum $x$-coordinate among the $5$ points.
    Use $y_{\max}$ to denote the maximum $y$-coordinate among the $5$ points and
    use $y_{\min}$ to denote the minimum $y$-coordinate among the $5$ points.
    There are at most $4$ points and at least $2$ points that contribute to  $x_{\max}$, $x_{\min}$, $y_{\max}$ and $y_{\min}$.
    Use $U_2$ to denote the set of the points that contribute to  $x_{\max}$, $x_{\min}$, $y_{\max}$ and $y_{\min}$.
    Obviously, $U_2\subsetneq S_2$. 
    Since $U_2$ is the set of points that contribute to  $x_{\max}$, $x_{\min}$, $y_{\max}$ and $y_{\min}$,
    if a 2-dimension axis-aligned rectangle contions all points in set $U_2$, it must contion all points in $S_2$.
    Then for set $U_2$, there is no such 2-dimension axis-aligned rectangle can satisfy the two constrains at the same time:
    \begin{enumerate}
        \item For each point in set $U$, the point is inside the axis-aligned rectangle.
        \item  For each point in set $S$ but not in set $U$, the point is not inside the axis-aligned rectangle.
    \end{enumerate}

    Thus, no $5$ points on the plane $\mathbb{R}^2$ can be shattered by the class $C$ of axis-aligned
    rectangles.
\end{proof}
\noindent \blue{
(b) Compute the VC-dimension of the class $C_k$ of $k$-dimension axis-aligned rectangles in $\mathbb{R}^k$.
In particular, you need to find a number $d$ such that there exist $d$ points in $\mathbb{R}^k$ that can be shattered by the $C_k$, and
prove that any $d+1$ points in $\mathbb{R}^k$ cannot be shattered by $C_k$.}\\
The VC-dimension of the class $C_k$ of $k$-dimension axis-aligned rectangles in $\mathbb{R}^k$ is $2k$.
\begin{proof}
    \noindent
    \begin{enumerate}
        \item  There exists $2k$ points in $\mathbb{R}^k$ that can be shattered by the $C_k$.
        
    A possible set of the $2k$ points is as follows and use $S^\prime_k$ to denote it.
    \begin{align}
        \nonumber S^\prime_k=&\{(1,0,0,...,0),(-1,0,0,...0),(0,1,0,...0),(0,-1,0,...,0),\\
        \nonumber &(0,0,1,...,0),(0,0,-1,...,0),...,(0,0,0,...,1),(0,0,0,...,-1)\}
    \end{align}

    Use $U^\prime_k$ to denote any subset of $S^\prime_k$.
    Since any point in $S^\prime_k$ can provide the minimum or maximum value in a dimension,
    for  each subset $U^\prime_k$ of $S^\prime_k$, we can always find a $k$-dimension axis-aligned rectangle such that 
    all points in $U^\prime_k$ are inside the $k$-dimension axis-aligned rectangle and 
    all points in set $S^\prime_k$ but not in set $U^\prime_k$ are not inside the $k$-dimension axis-aligned rectangle.
    Thus we have that there exist $2k$ points in $\mathbb{R}^k$ that can be shattered by the $C_k$.
    \item Any $2k+1$ points in $\mathbb{R}^k$ cannot be shattered by $C_k$.

    Use $S_k$ to denote any set of $2k+1$ distinct points in $\mathbb{R}^k$.
    Use set $U_k$ to denote the set of points that can contribute the maximum or minimum value in each dimension among points in set $S_k$.
    Obviously, $|U_k|\in[2,2k]$ and $U_k\subsetneq S_k$.
    Since $U_k$ is the set of points that can contribute the maximum or minimum value in each dimension,
    if a $k$-dimension axis-aligned rectangle contions all points in set $U_k$, it must contion all points in $S_k$.
    Then for set $U_k$, there is no such $k$-dimension axis-aligned rectangle can satisfy the two constrains at the same time:
    \begin{enumerate}
        \item For each point in set $U_k$, the point is inside the $k$-dimension axis-aligned rectangle.
        \item  For each point in set $S_k$ but not in set $U_k$, the point is not inside the $k$-dimension axis-aligned rectangle.
    \end{enumerate}

    Thus, no $2k+1$ points in $\mathbb{R}^k$ can be shattered by the class $C_k$ of $k$-dimension axis-aligned rectangles.
\end{enumerate}
   
By definition, the VC-dimension of $(X, C)$ is the maximum cardinality of a subset $S \subseteq X$ that is shattered by $C$.
Hence, in this problem, the VC-dimension of the class $C_k$ of $k$-dimension axis-aligned rectangles in $\mathbb{R}^k$ is $2k$.
\end{proof}
\noindent \blue{2. \textbf{Conditional Expectation.}\\
Suppose $Y:\Omega\rightarrow\mathbb{R}$ is a random variable and
$W:\Omega\rightarrow\mathcal{U}$ is a random object 
defined on the same probability space $(\Omega, \mathcal{F},Pr)$.
Prove that $\mathbb{E}[Y]=\mathbb{E}[\mathbb{E}[Y|W]]$.
You may assume that both $\Omega$ and $\mathcal{U}$ are finite.
}

$\mathbb{E}[Y|W]:\mathcal{U}\rightarrow \mathbb{R}$ is a function that for $u\in \mathcal{U}$, $\mathbb{E}[Y|W](u):=\mathbb{E}[Y|W=u]$.

The expected value of the random variable $\mathbb{E}[Y|W]$ is a function
of $W$ and takes on the value $\mathbb{E}[Y|W=u]$ when $W=u$.
Then,
\begin{align}
    \nonumber \mathbb{E}[\mathbb{E}[Y|W]]=\sum_u{\mathbb{E}[Y|W=u]\Pr[W=u]}
\end{align}

Since $\mathbb{E}[Y]=\sum_u \mathbb{E}[Y|W=u]\Pr[W=u]$, we have $\mathbb{E}[Y]=\mathbb{E}[\mathbb{E}[Y|W]]$.\\
\blue{
3.\textbf{Using $\epsilon$-Net for learning.} Suppose $X$ is a set with some underlying distribution $D$ 
and $C$ is a class of boolean functions on $X$, and the VC-dimension of $(X, C)$ is $d$.
Moreover, suppose there is some function $f_0\in C$ that corresponds to some classifier that we wish to learn.  
The model we have is that we can sample a random $x\in X$ and ask for the value $f_0(x)$.
After seeing $m$ such samples $S$ in $X$, we pick a function $f_1\in C$ that agrees with $f_0$ on $S$.
The hope is that $f_1$ and $f_0$ would agree on most points in $X$ (according to distribution $D$).}
\\\blue{(a) Define another class $C^{\prime}$ of boolean functions on $X$ such that if $S$ is an $\epsilon$-net under $C^{\prime}$, 
and $f\in C$ is a function that disagrees with $f_0$ on more than $\epsilon$ fraction 
(weighted according to $D$) of points in $X$, then there exists some $x\in S$ such that $f(x)\neq f_0(x)$.
Prove  the VC-dimension of $(X,C^\prime)$ for the class $C^{\prime}$ that you have constructed. 
}\\
Let $F$ be a class of boolean functions on $X$ satisfies that 
\begin{enumerate}
    \item $F\subset C$
    \item  For each $f\in F$, $f$ is a function that disagrees with $f_0$ on more than $\epsilon$ fraction of points in $X$.
    \item  For each $f\in C\setminus F$, $f$ is not a function that disagrees with $f_0$ on more than $\epsilon$ fraction of points in $X$.
\end{enumerate}
Construct $C^{\prime}$ according to $F$ as follows.
Initially, set $C^{\prime}=\emptyset$.
For each $f\in F$, construct a new boolean function $f^{\prime}(x)=f(x)\oplus f_0(x)$ and $C^{\prime}=C^{\prime}\cup f^{\prime}$.
\\\textbf{Correctness: }Since $S$ is an $\epsilon$-net under $C^{\prime}$ and for each $f^{\prime}\in C^{\prime}$, $\mathbb{E}_X[f^{\prime}]\ge \epsilon$, by the definition of $\epsilon$-net, we have there exists $x\in S$ such that $f^{\prime}=1$.
Since $f^{\prime}(x)=f(x)\oplus f_0(x)$, we have that there exists some $x\in S$ such that $f(x)\neq f_0(x)$.
\\
~\\The VC-dimension of $(X, C^{\prime})$ is $d$.
\begin{proof}
    Since the VC-dimension of $(X, C)$ is $d$, by the definition of VC-dimension, we have that:  
    \begin{enumerate}
        \item There exists a subset $S_d$ such that $|S_d|=d$ and for all subsets $U_d$ of $S_d$, there exists $f\in C$ such that for all $x\in U_d$, $f(x)=1$ and for all $x\in S_d\setminus U_d$, $f(x)=0$.
        \item There is not a subset $S_{d+1}$ such that $|S_{d+1}|=d+1$ and for all subsets $U_{d+1}$ of $S_{d+1}$, there exists $f\in C$ such that for all $x\in U_{d+1}$, $f(x)=1$ and for all $x\in S_{d+1}\setminus U_{d+1}$, $f(x)=0$.
    \end{enumerate}
    The relationship between $f(x)$, $f_0(x)$ and $f^{\prime}(x)$ is that if $f(x)==f_0(x)$, $f^{\prime}(x)=0$; otherwise $f^{\prime}(x)=1$.
    Order all the elements in $S_d$ in any order and denote the elements as $x_1,x_2,x_3,\cdots,x_d$.
    For any subset of $S_d$, $U_d$, there exists  $f\in C$ such that for all $x\in U_d$, $f(x)=1$ and for all $x\in S_d\setminus U_d$, $f(x)=0$, 
    In another word, there exists $2^d$ $f$ such that 
    \begin{align}
        \nonumber f^0(x_1),f^0(x_2),f^0(x_3),\cdots, f^0(x_d)&=&0,0,0,...0\\
        \nonumber f^1(x_1),f^1(x_2),f^1(x_3),\cdots, f^1(x_d)&=&1,0,0,...0\\
        \nonumber f^2(x_1),f^2(x_2),f^2(x_3),\cdots, f^2(x_d)&=&0,1,0,...0\\
        \nonumber \cdots &=& \cdots\\
        \nonumber f^{2^d-1}(x_1),f^{2^d-1}(x_2),f^{2^d-1}(x_3),\cdots, f^{2^d-1}(x_d)&=&1,1,1,...1
    \end{align}
    For any $f_0(x_i)$, since the fact that if $f(x_i)==f_0(x_i)$, $f^{\prime}(x_i)=0$; otherwise $f^{\prime}(x_i)=1$,
    we have that there also exists $2^d$ $f^{\prime}$ such that 
    \begin{align}
        \nonumber f^{\prime 0}(x_1),f^{\prime 0}(x_2),f^{\prime 0}(x_3),\cdots, f^{\prime 0}(x_d)&=&0,0,0,...0\\
        \nonumber f^{\prime 1}(x_1),f^{\prime 1}(x_2),f^{\prime 1}(x_3),\cdots, f^{\prime 1}(x_d)&=&1,0,0,...0\\
        \nonumber f^{\prime 2}(x_1),f^{\prime 2}(x_2),f^{\prime 2}(x_3),\cdots, f^{\prime 2}(x_d)&=&0,1,0,...0\\
        \nonumber \cdots &=& \cdots\\
        \nonumber f^{\prime 2^d-1}(x_1),f^{\prime 2^d-1}(x_2),f^{\prime 2^d-1}(x_3),\cdots, f^{\prime 2^d-1}(x_d)&=&1,1,1,...1
    \end{align}
    We have that for set $S_d$, for all subsets $U_d$ of $S_d$, there exists $f^{\prime} \in C$ such that for all $x\in U_d$, $f^{\prime}(x) =1$ and for all $x\in S_d\setminus U_d$, $f^{\prime}(x)=0$.
    By using the same method, we have that there is not a subset $S_{d+1}$ such that $|S_{d+1}|=d+1$ and for all subsets $U_{d+1}$ of $S_{d+1}$, there exists $f^{\prime}\in C$ such that for all $x\in U_{d+1}$, $f^{\prime}(x)=1$ and for all $x\in S_{d+1}\setminus U_{d+1}$, $f^{\prime}(x)=0$.
    Therefore, we have the result that the VC-dimension of $(X, C^{\prime})$ is $d$.
\end{proof}
\noindent\blue{(b) How many samples are enough such that with probability at least $1-\delta$ the function $f_1$ returned disagrees with $f_0$ on at most $\epsilon$ weighted fraction of points in $X$?}
\\
Construct $C^{\prime}$ as above and by (a), we have the VC-dimension of $(X, C^{\prime})$ is $d$.
The problem is transformed into finding the minimum sample times $m$ and construct $S$ such that $S$ is an 
$\epsilon$-net under $C^{\prime}$ with probability at least $1-\delta$.
The probability that $S$ is not an $\epsilon$-net after sampling $m$ times is at most $\delta$.
Let $C^{\prime}_\epsilon:=\{f\in C^{\prime}: E_x[f]\ge \epsilon\}$.
For $f\in C^{\prime}_\epsilon$, the probability that a point sampled form $X$ would be labelled $1$ is at least $\epsilon$.
The failure probability that all points in $S$ are labelled $0$ under $S$ is at most $(1-\epsilon)^m\le e^{-\epsilon m}$. 
The probability that the set $S$ fails for some $f\in C^{\prime}_\epsilon$ is at most $|C^{\prime}_\epsilon|e^{-\epsilon m}\le |C^{\prime}|e^{-\epsilon m}\le \delta$, which implies $m\le \frac{1}{\epsilon}(\ln |C^{\prime}|+\ln \frac{1}{\delta})$.
$f_1$ is constructed as follows, after sampling $m$ times from $X$ and construct $S$, pick a function $f_1$ from $C$ such that for all $x\in S$, $f_1(x)\neq f_0(x)$.

